\documentclass[leqno,hidelinks]{article}
\usepackage[german,english]{babel}
\usepackage{amsmath,amssymb,amsthm}
\usepackage{color,hyperref}
\usepackage{enumerate}
\usepackage{microtype}

\linespread{1.25}

\definecolor{darkblue}{rgb}{0,0,.7}
\definecolor{ultrav}{rgb}{0.5,0,1}

\hypersetup{colorlinks,
            breaklinks,
            linkcolor=ultrav,
            urlcolor=darkblue,
            anchorcolor=darkblue,
            citecolor=darkblue}

\newtheoremstyle{}
        {}%space above
        {}%space below
        {}%body font
        {}%Indent amount
        {\bfseries}%Theorem head font
        {}%Punctuation after theorem head
        {.5em}%Space after theorem head
        {#1.#2.~\thmnote{#3.}
}%Theorem head spec
\swapnumbers

\theoremstyle{definition}

\newtheorem{satz}{\protect\satzname}
\newtheorem{deff}[satz]{\protect\deffname}
\newtheorem*{zusatz}{\protect\zusatzname}
\newtheorem*{hilfssatz}{\protect\hilfssatzname}

\newcommand{\satzname}{}
\newcommand{\deffname}{}
\newcommand{\zusatzname}{}
\newcommand{\hilfssatzname}{}

\addto\captionsgerman{%
    \renewcommand{\satzname}{\hspace{-4pt}.\ Satz}%
    \renewcommand{\deffname}{\hspace{-4pt}.\ Definition}%
    \renewcommand{\zusatzname}{Zusatz}%
    \renewcommand{\hilfssatzname}{Hilfssatz}%
}

\addto\captionsenglish{%
    \renewcommand{\satzname}{\hspace{-4pt}.\ Theorem}%
    \renewcommand{\deffname}{\hspace{-4pt}.\ Definition}%
    \renewcommand{\zusatzname}{Corollary}%
    \renewcommand{\hilfssatzname}{Lemma}%
}

\newcommand\Beweis{\medskip \newline $ \phantom{'.'} \rhd \ $}%
\newcommand\beweis{ $ \phantom{'.'} \rhd \ $}%

% From https://narkive.com/P4sNM0k5:7.443.63
\newcommand\TeilVon{\mathrel{\raisebox{0.45ex}{$\mathfrak{3}$}}}
\newcommand{\partof}{\subseteq}

% \pagestyle{myheadings}
% \markright{Dedekind, Second definition of the finite and infinite\hfill}

% \usepackage[landscape, total={8in, 8in}]{geometry}
\usepackage[total={8in, 10in}]{geometry}

% \usepackage{showframe}
% \setcounter{page}{451}

% \newcommand{\sref}[1]{(\underline{\ref{#1}})}% with ()
\newcommand{\sref}[1]{\underline{\ref{#1}}}% without ()

\usepackage{paracol}

\begin{document}
\hypersetup{pageanchor=false}

\title{Zweite Definition des Endlichen und Unendlichen.}
\author{Richard Dedekind}
\date{1889. 3. 9. \phantom{} [ 9th March 1889 ]}
\maketitle

\thispagestyle{empty}

\begin{paracol}{2} % specify two columns...

\selectlanguage{german}

\noindent Zuerst veröffentlicht in der zweiten Auflage (1893) der Schrift
\textit{\glqq Was sind und was sollen die Zahlen?\grqq{}} Seite XVII, in der Form:%

\begin{quote}
Ein System $S$ heißt endlich, wenn es sich so in sich selbst abbilden
lässt, dass kein echter Teil von $S$ in sich selbst abgebildet
wird; im entgegengesetzten Fall heißt $S$ ein unendliches System.
\end{quote}

Verfolgung dieser Definition eines endlichen Systems $S$
\emph{ohne Benutzung der natürlichen Zahlen}.

Es sei $\varphi$ eine Abbildung von $S$ in sich selbst, durch welche kein echter
Teil von $S$ in sich selbst abgebildet wird.
Kleine lateinische Buchstaben  $a, b \ldots z$ bedeuten immer \emph{Elemente} von
$S$, große lateinische Buchstaben $A, B \ldots Z$ bedeuten \emph{Teile} von $S$;
die durch $\varphi$ erzeugten Bilder von $a, A$ werden resp. mit $a', A'$ bezeichnet.

Dass $A$ \emph{Teil von} $B$ ist, wird durch $A \TeilVon B$ ausgedrückt. Das aus den
Elementen $a, b, c, \ldots $ bestehende System wird mit $[a, b, c \ldots]$ bezeichnet.

Es ist also
\begin{equation}\label{geq1}
				S' \TeilVon S
\end{equation}
und% \marginpar{\textit{Die Definition des Endlichen.}}
\begin{equation}\label{geq2}
		\text{aus } A' \TeilVon A \text{ folgt } A = S.
\end{equation}

\begin{satz}\label{gthm1}$S' = S$.
\Beweis
Jedes Element von $S$ ist Bild von (mindestens) einem Element $r$ von $S$. Denn
aus \eqref{geq1} folgt $(S')' \TeilVon S'$, also nach \eqref{geq2} unser Satz.
\end{satz}
Jedes aus einem einzigen Element $s$ bestehende System $[s]$ ist endlich, weil
es keinen echten Teil besitzt und durch die identische Abbildung in sich selbst
abgebildet wird. Dieser Fall wird im folgenden \emph{ausgeschlossen}, $S$ bedeutet
ein endliches System, das \emph{nicht} aus einem einzigen Element besteht.

\begin{satz}\label{gthm2}
Jedes Element $s$ ist verschieden von seinem Bilde $s'$, in Zeichen: $s \neq s'$.
\Beweis
Denn wäre $s = s'$, so wäre $[s]' = [s'] = [s] \TeilVon [s]$, nach \eqref{geq2}
auch $[s] = S$ im Widerspruch zu unserer Annahme über $S$.
\end{satz}

\newpage

\begin{deff}\label{gdef3}
Ist $s$ ein bestimmtes Element von $S$ so soll mit $H_s$ jeder solche Teil von
$S$ bezeichnet werden, der den beiden folgenden Bedingungen genügt:
\begin{enumerate}[I.]
\item $s$ ist Element von $H_s$, also $[s] \TeilVon H_s$, also auch
\[
	[s] + H_s = H_s.
\]
\item Ist $h$ ein von $s$ verschiedenes Element von $H_s$, so ist auch $h'$
Element von $H_s$; ist also $H \TeilVon H_s$, aber $s$ nicht in $H$ enthalten,
so ist $H' \TeilVon H_s$.
\end{enumerate}
\end{deff}

\begin{satz}\label{gthm4}
$S$ und $[s]$ sind spezielle Systeme $H_s$, und $[s]$ ist der Durchschnitt
(die Gemeinheit) aller dem Elemente $s$ entsprechenden Systeme $H_s$.
\Beweis Offenbar.
\end{satz}

\begin{satz}\label{gthm5}
$H_s = S$ oder \textit{echter} Teil von $S$, je nachdem $s'$ in $H_s$ liegt oder nicht.
\Beweis
Denn wenn $s'$ in $H_s$ liegt, so folgt aus (\sref{gdef3}{\color{ultrav}.II)},
dass $H'_s \TeilVon H_s$, also nach \eqref{geq2}, dass $H_s = S$ ist; und umgekehrt,
wenn $H_s = S$, so liegt auch $s'$ in $H_s$.
\end{satz}

\begin{satz}\label{gthm6}
Ist $H_s$ \emph{echter} Teil von $S$, so ist $s'$ das einzige Element von $H'_s$,
das außerhalb $H_s$ liegt.
\Beweis
Denn jedes Element $k$ von $H'_s$ ist Bild $h'$ von mindestens einem Element $h$
in $H_s$; ist nun $k=h'$ verschieden von $s'$, so ist auch $h$ verschieden von $s$,
und folglich (nach \sref{gdef3}{\color{ultrav}.II)} liegt $k = h'$ in $H_s$,
während das Element $s'$ von $H'_s$ nach \sref{gthm5} außerhalb $H_s$ liegt.
\end{satz}

\begin{satz}\label{gthm7}
Jedes System $H'_s$ ist ein System $H_{s'}$, das heißt (Definition \sref{gdef3}):
\begin{enumerate}[I'.]
    \item $s'$ ist Element von $H'_s$.
    \item Ist $k$ ein von $s'$ verschiedenes Element von $H'_s$, so liegt auch $k'$ in $H'_s$.
\end{enumerate}
\beweis
Das Erste folgt daraus, dass $s$ in $H_s$ liegt, das Zweite daraus,
dass nach Satz \sref{gthm6} $k$ in $H_s$ liegt.
\end{satz}

\begin{satz}\label{gthm8}
Sind $A$, $B$, $C$ \ldots\ spezielle, demselben $s$ entsprechende Systeme $H_s$,
so ist auch ihr Durchschnitt $H$ ein System $H_s$.
\Beweis
Denn zufolge \sref{gdef3}{\color{ultrav}.I} ist $s$ gemeinsames Element von
$A, B, C, \ldots$, also auch Element von $H$. Ist ferner $h$ ein von $s$
verschiedenes Element von $H$, so ist zufolge \sref{gdef3}{\color{ultrav}.II} das
Bild $h'$ Element von $A$, von $B$, von $C$, \ldots, also auch von $H$. Mithin
erfüllt $H$ die beiden für jedes $H_s$ charakteristischen Bedingungen I, II
in \sref{gdef3}.
\end{satz}

\begin{deff}\label{gdef9}
Sind $a$, $b$ bestimmte Elemente von $S$, so soll das Symbol $ab$ den Durchschnitt
aller derjenigen Systeme $H_b$ bedeuten (\emph{Strecke} $ab$), welche (wie z. B. $S$)
das Element $a$ enthalten.
\end{deff}

\newpage

\begin{satz}\label{gthm10}
$a$ ist Element von $ab$, d. h. $[a] \TeilVon ab$.
\Beweis
Denn $ab$ ist der Durchschnitt von lauter solchen Systemen $H_b$ in denen $a$
liegt. (a \emph{Anfang} von $ab$.)
\end{satz}

\begin{satz}\label{gthm11}
$ab$ ist ein System $H_b$, d. h. $[b] \TeilVon ab$, und wenn $s$ ein von $b$
verschiedenes Element von $ab$ ist, so ist $[s'] \TeilVon ab$.
\Beweis
Dies folgt aus \sref{gthm8}.
\end{satz}

Also $b$ Element (\emph{Ende}) von $ab$. Ist $H \TeilVon ab$, aber $b$ nicht in
$H$ enthalten, so ist $H' \TeilVon ab$.

\begin{satz}\label{gthm12}
Aus $[a] \TeilVon H_b$ folgt $ab \TeilVon H_b$.
\Beweis
Unmittelbare Folge von \sref{gdef9}.
\end{satz}

\begin{satz}\label{gthm13}
$aa = [a]$.
\Beweis
Dies folgt aus \sref{gthm4}, weil $aa$ der Durchschnitt aller $H_a$  ist, die
ja alle das Element $a$ enthalten nach \sref{gdef3}.I
\end{satz}

\begin{satz}\label{gthm14}
Ist $b'$ Element von $ab$, so ist $ab = S$.
\Beweis
Dies folgt aus \sref{gthm11} und \sref{gthm5}.
\end{satz}

\begin{satz}\label{gthm15}
$b' b = S$.
\Beweis
Dies folgt aus \sref{gthm14} und \sref{gthm10}.
\end{satz}

\begin{satz}\label{gthm16}
Ist $c$ Element von $ab$, so ist $cb \TeilVon ab$.
\Beweis
Dies folgt aus \sref{gthm12}, denn $ab$ ist ein $H_b$,
(nach \sref{gthm11}), welches das Element $c$ enthält.
\end{satz}

\begin{satz}\label{gthm17}
Bedeutet $A + B$ das aus $A, B$ zusammengesetzte System, so ist
\[
	a'b + b'a = S.
\]%
\beweis
Denn wenn $s$ Element von $a'b$, so ist $s'$ in $b'a$ oder $a'b$
enthalten, je nachdem $s = b$ oder verschieden von $b$ (zufolge \sref{gthm10}
oder ~\sref{gthm11} und \sref{gdef3}.II), und ebenso, wenn $s$ Element von
$b'a$, so ist $s'$ in $a'b$ oder $b'a$ enthalten; also ist
$(a'b + b'a)' \TeilVon a'b + b'a$; hieraus folgt der Satz nach \eqref{geq2}.
\end{satz}

\begin{satz}\label{gthm18}
Ist $a$ verschieden von $b$, so ist $ab = [a] + a'b$.
\Beweis
Denn da $a$ ein von $b$ verschiedenes Element von $ab$ ist, so ist $a'$ Element
von $ab$ \sref{gthm10}, \sref{gthm11}, und folglich \sref{gthm16} ist
$a'b \TeilVon ab$; da ferner \sref{gthm10} auch $[a] \TeilVon ab$, mithin
\[
	[a] + a'b \TeilVon ab.
\]
Ferner: jedes von $b$ verschiedene Element $s$ von $[a] + a'b$ ist entweder
$= a$ oder ein von $b$ verschiedenes Element von $a'b$, in beiden Fällen ist
$s'$ (nach \sref{gthm10}, \sref{gthm11}) Element von $a'b$, also auch von
$[a]+ a'b$, und da \sref{gthm11} auch $[b] \TeilVon [a] + a'b$, so ist
$[a] + a'b$ ein System $H_b$; da endlich auch $[a] \TeilVon [a] + a'b$, so ist
\sref{gthm12} auch
\[
	ab \TeilVon [a] + a' b.
\]
Aus der Vergleichung beider Resultate folgt der Satz.
\end{satz}

\newpage

\begin{satz}\label{gthm19}
Sind $a, b$ verschiedene Elemente von $S$, so liegt $a$ außerhalb $a'b$,
und $b$ liegt außerhalb $b'a$.
\Beweis
Nimmt man nämlich das Gegenteil an, es gebe ein von $b$ verschiedenes
Element $a$, das in $a'b$ liegt, und bezeichnet mit $A$ das System aller
solcher Elemente $a$, so ergibt sich folgendes.

Setzt man $a' = s$, so liegt $a$ in $sb$, und da $a$ verschieden von $b$ ist, also
(nach \sref{gthm13}) nicht in $bb$ liegt, so ist $s$ verschieden von $b$, und
hieraus folgt (nach \sref{gthm18}), dass $sb = [s] + s'b$ ist. Da ferner $a$
(nach \sref{gthm2}) verschieden von $s$ ist und in $sb$ liegt, so muss $a$ in
$s'b$ liegen, und hieraus folgt (wieder nach \sref{gthm1}), dass auch $s$
(als Bild $a'$) in $s'b$ liegt.

Mithin ist das Bild $a'$ eines jeden Elementes $a$ von $A$ ebenfalls in $A$
enthalten, also $A' \TeilVon A$. Da aber hieraus $a = S$ folgen würde, während
doch $A$ das Element $b$ nicht enthält, so ist unsere Annahme unzulässig, also
der Satz wahr, w.z.b.w.

Der zweite Teil folgt durch Vertauschung von $a$ mit $b$.%
\end{satz}

\begin{satz}\label{gthm20}
Sind $a, b$ verschieden, so haben die Strecken $a'b, b'a$ kein gemeinsames
Element.
\Beweis
Nimmt man nämlich das Gegenteil an, es gebe ein gemeinsames Element
$m\,$ von $a'b\,$, $b'a\,$, so folgt aus dem vorhergehenden Satz
\sref{gthm19}, dass $m$ verschieden von $b$ und von $a$ ist; mithin muss
\sref{gthm11} das Bild $m'$ ebenfalls gemeinsames Element von $a'b$ und $b'a$ sein.

Bezeichnet man daher mit $M$ das System aller solcher Elemente $m$, so ist
$M' \TeilVon M$, also $M = S$. Dies ist aber unmöglich, weil $a, b$ Elemente von
$S$, aber nicht Elemente von $M$ sind. Also ist unser Satz wahr.
\end{satz}

\begin{satz}\label{gthm21}
Sind $a, b$ verschieden, so sind auch die Bilder $a', b'$ verschieden.
\Beweis
Denn sonst hätten die Strecken $a'b, b'a$ ein gemeinsames Element $a' = b'$, weil
$a'$ (nach \sref{thm10}) Element von $a'b$ und $b'$ Element von $b'a$ ist.
\end{satz}

\begin{satz}\label{gthm22}
Aus $cb = S$ folgt $c = b'$.
\Beweis
Es gibt (nach \sref{thm1} und \sref{thm21}) in $S$ ein und nur ein Element $a$,
welches der Bedingung $a' = c$ genügt, und es ist also $a'b = S$, mithin
$[a] \TeilVon a'b$; es muss daher (\sref{gthm19}) $a = b$, also $c = b'$ sein, w.z.b.w.
\end{satz}

\begin{satz}\label{gthm23}
Sind $a, b$ verschieden, so ist jedes Element von $S$ in einer und nur einer
der Strecken $a'b, b'a$ enthalten.
\Beweis
Dies folgt aus \sref{thm17} und \sref{thm20}.
\end{satz}

\begin{satz}\label{gthm24}
Sind $a,\, b,\, c$ verschieden, so haben die Strecken $b'c,\, c'a$, $a'b$ kein
gemeinsames Element, und dasselbe gilt von den Strecken $a'c,\, b'a,\, c'b$.
\Beweis
Denn die gegenteilige Annahme, es gebe ein den Strecken $b'c,\, c'a,\, a'b$
gemeinsames Element $m$, führt zu einem Widerspruch. Es sei $M$ das System aller
solcher Elemente. Da (nach \sref{gthm19}) $a$ nicht in $a'b$, $ b$ nicht in
$b'c$, $c$ nicht in $c'a$ liegt, so ist $m$ verschieden von $c, a, b$, und folglich
(\sref{thm11}) ist $m'$ ebenfalls gemeinsames Element von $b'c,\, c'a,\, a'b$,
also Element von $M$.

Mithin ist $M' \TeilVon M$, also $M = S$. Dies ist aber unmöglich, weil $M$ keins
der Elemente $a,\, b,\, c$ enthält. Also ist unser Satz wahr.

Der zweite Teil ergibt sich aus dem ersten, wenn man $a$ mit $b$ vertauscht,
wodurch die Annahme nicht geändert wird.
\end{satz}

\begin{zusatz}\label{zusatz1}
Setzt man (wie auch in dem folgenden \sref{gthm25}):
\[
    A = c'b,\ B = a'c,\ C = b'a;\  A_1 = b'c,\ B_1 = c'a,\ C_1 = a'b,
\]
so ist $A - B - C = 0$ (leer) (dabei bedeutet das Zeichen $-$ den Durchschnitt)
 und $A_1 - B_1 - C_1 = 0$ (leer) und (nach \sref{gthm17}, \sref{gthm20}) ist
\begin{gather*}
    S = A + A = B + B_1 = C + C_1; \\
    \,0 = A - A_1 = B - B_1 = C - C_1.
\end{gather*}
Dies gilt auch dann (nach \sref{gthm20}), wenn von den Elementen $a$, $b$, $c$
wenigstens zwei verschieden sind.
\end{zusatz}

\begin{satz}\label{gthm25}
Sind $a$, $b$, $c$ verschieden, so tritt einer und nur einer der beiden
folgenden Fälle ein: Entweder ist
\begin{gather*}
    b'c = b'a + a'c, \quad c'a = c'b + b'a, \quad a'b = a'c + c'b \\
    c'b = c'a - a'b, \quad a'c = a'b - b'c, \quad b'a = b'c - c'a
\end{gather*}
und jedes Element von $S$ liegt in einer, aber nur einer der Strecken
$c'b$, $a'c$, $b'a$; oder es ist
\begin{gather*}
    c'b = c'a + a'b, \quad a'c = a'b + b'c, \quad b'a = b'c + c'a \\
    b'c = b'a - a'c, \quad c'a = c'b - b'a, \quad a'b = a'c - c'b
\end{gather*}
und jedes Element von $S$ liegt in einer, aber nur einer der Strecken
$b'c$, $c'a$, $a'b$.
\Beweis
Zufolge \sref{gthm23} liegt $c$ entweder in $a'b$ oder in $b'a$. Wir betrachten
nur den ersten Fall, weil aus ihm der zweite durch Vertauschung von $a$ mit $b$
hervorgeht. Da $c$ in $a'b$ liegt und von $b$ verschieden ist, so liegt
(nach \sref{gthm11})) auch $c'$ in $a'b$, und folglich (\sref{gthm16}) ist
$c'b \TeilVon a'b$; hieraus folgt (nach \sref{gthm19}), dass $c'b$ mit $b'a$ kein
gemeinsames Element hat; nun ist (mit \sref{gthm17}) $a'b + b'a = b'c + c'b$,
mithin $b'a \TeilVon b'c$ und folglich (\sref{gthm11}) liegt $a$ in $b'c$.

Aus der Annahme, dass $c$ in $a'b$ liegt, hat sich also ergeben: $c'b \TeilVon a'b$,
$b'a \TeilVon b'c$, $a$ liegt in $b'c$. Auf dieselbe Weise ergeben sich aus
dieser letzten Folgerung, wenn man $c$, $a$, $b$ in der Annahme resp. durch $a$,
$b$, $c$ ersetzt, wieder die Folgerungen $a'c \TeilVon b'c$, $c'b \TeilVon c'a$,
$b$ liegt in $c'a$; und hieraus folgt abermals $b'a \TeilVon c'a$,
$a'c \TeilVon a'b$ (und die erste Annahme: $c$ liegt in $a'b$).

Es ist also: $c'b \TeilVon a'b$, $b'a \TeilVon b'c$, $a'c \TeilVon b'c$,
$c'b \TeilVon c'a$, $b'a \TeilVon c'a$, $a'c \TeilVon a'b$, also auch
$b'a + a'c \TeilVon b'c$, $c'b + b'a \TeilVon c'a$, $a'c + c'b \TeilVon a'b$.

\newpage

Läge nun z. B. ein Element von $b'c$ weder in $b'a$, noch in $a'c$,
so wäre es (nach \sref{gthm23}) gemeinsames Element von $b'c$, $a'b$, $c'a$, was
(nach \sref{gthm24}) unmöglich ist; mithin ist $b'c \TeilVon b'a + a'c$, also
auch $b'c = b'a + a'c$, und ebenso folgt $c'a = c'b + b'a$, $a'b = a'c + c'b$.

Hätten nun z. B. $b'a$, $a'c$ ein gemeinsames Element, so wäre dasselbe
auch gemeinsames Element von $b'c$, $c'a$, $a'b$, was (nach \sref{gthm24})
nicht der Fall ist. Aus $S = b'c + c'b$ folgt endlich $S = b'a + a'c + c'b$,
womit unser Satz vollständig bewiesen ist.
\end{satz}

\begin{zusatz}\label{zusatz2}
Es kann nie gleichzeitig $[a] \TeilVon cb$ und $[b] \TeilVon ca$ sein;
weil (nach \sref{gthm18}) dann auch gleichzeitig $[a] \TeilVon c'b$ und
$[b] \TeilVon c'a$  sein müsste, was unmöglich.
\end{zusatz}

\begin{satz}\label{gthm26}
Aus $ab = cb$ folgt $a = c$, und wenn $ab = cd$ ein echter Teil von $S$ ist,
so ist $a = c$, $b = d$.
\Beweis
Dies folgt schon aus früheren Sätzen. Da (nach \sref{gthm10}) $c$ in $cb$, also
auch in $ab$ liegt, so muss, falls $a = b$, also $ab = [a]$ ist, auch $c = a$ sein.
Ist aber $a$ verschieden von $b$, so ist (nach \sref{gthm18}) $ab = [a] + a'b$, und
(nach \sref{gthm19}) $a'b$ ist echter Teil von $ab$; nimmt man an, es sei $c$
verschieden von $a$, so muss $c$ in $a'b$ liegen, also ist (nach \sref{gthm16})
$cb \TeilVon a'b$, also $cb$ echter Teil von $ab$; da aber $cb = ab$ ist, so ist
diese Annahme unzulässig, mithin immer $c = a$, w.z.b.w.

Ist ferner $ab = cd$ ein echter Teil von $S$, so muss $b = d$ sein; ist nämlich
$b$ verschieden von $d$, so muss (\sref{gthm11}) auch $b'$ in $cd$, also auch in
$ab$ liegen; dann wäre aber (\sref{gthm14}) $ab = S$ gegen die Voraussetzung,
also ist $b = d$, mithin $ ab = cb$, also auch $a = c$, w.z.b.w.
\end{satz}

\begin{satz}\label{gthm27}
Jedes (in \sref{gdef3} erklärte) System $H_s$, ist eine Strecke $a's$ mit dem
Ende $s$ und ihr Anfang $a'$ ist völlig bestimmt.
\Beweis
Ist $H_s= S$, so ist $H_s= s's$ (nach \sref{gthm15}). Ist $H_s$ aber echter Teil
von $S$, so sei $A$ das System aller außerhalb $H_s$ liegenden Elemente von $S$,
also $S = A + H_s$. Da $A$ echter Teil von $S$ ist, so kann nicht $A' \TeilVon A$
sein, es gibt also gewiss ein Element $a$ in $A$, dessen Bild $a'$ außerhalb $A$,
also in $H_s$ liegt; da (nach \sref{gthm12}) folglich $a's \TeilVon H_s$ ist, so
haben $a's$, $A$ kein gemeinsames Element.

Da $a$ in  $A$, $s$ in $H_s$ (sogar in $a's$) liegt, so sind $a$, $s$ verschieden,
also haben (nach \sref{gthm20}) die Strecken $a's$, $s'a$ kein gemeinsames Element,
und (nach \sref{gthm17}) ist  $a's + s'a = S = H_s + A$, mithin $A \TeilVon s'a$.
Nimmt man nun an, es sei $a's$ ein echter Teil von $H_s$, und bezeichnet mit $H$
das System aller derjenigen Elemente von $H_s$, welche außerhalb $a's$, also in
$s'a$, so ist $H_s = H + a's$, und $s'a = H + A$, also ist $H = H_s - s'a$ der
Durchschnitt der Systeme $H_s$, $s'a$.

Da nun weder $s$ noch $a$ in $H$ liegt, so folgt aus $H \TeilVon H_s$, und
$H \TeilVon s'a$ (nach \sref{gdef3} und \sref{gthm11}), dass auch $H' \TeilVon H_s$
und $H' \TeilVon s'a$, also auch $H' \TeilVon H$, mithin $H = S$ ist. Dies ist
aber unmöglich, weil $s$ (und ebenso $a$) außerhalb $H$ liegt. Mithin ist gewiss
$H_s= a's$, und $A = s'a$, w.z.b.w.
\end{satz}

\newpage

\begin{satz}\label{gthm28}
Der Durchschnitt von solchen Strecken $as$, $bs$, $\ldots$ welche dasselbe Ende
$s$ haben, ist selbst eine solche Strecke $hs$, und ihr Anfang $h$ ist
vollständig bestimmt.
\Beweis
Denn jede solche Strecke ist (nach \sref{gthm11}) ein System $H_s$, und (nach \sref{gthm8})
gilt dasselbe von ihrem Durchschnitt, woraus der Satz (nach \sref{gthm27}) folgt.
\end{satz}

\begin{zusatz}\label{zusatz3}
Der Durchschnitt der Strecken $as$, $bs$, $cs$, ... ist selbst eine dieser Strecken.
\end{zusatz}

Zum Beweise schicke man folgenden Hilfssatz voraus:

\begin{hilfssatz}\label{hilfssatz1}
Ist $hs$ echter Teil von $as$, und $k$ das Element, dessen Bild $k' = h$ ist,
so ist $hs$ auch echter Teil von $ks$, und zugleich ist $ks \TeilVon as$.
\Beweis
Wäre $k = s$, so wäre $hs = s's = S$, während doch $hs$ echter Teil von $as$,
also auch von $S$ ist. Da also $k$ verschieden von $s$ ist, so ist (nach \sref{gthm18})
$ks = [k] + hs$ und (nach \sref{gthm19}) $k$ nicht in $hs$ enthalten, also $hs$
echter Teil von $ks$. Da $hs$ echter Teil von $as$ ist, so sei $as = M + hs$,
wo $M$ das System aller Elemente $m$ von $as$, die außerhalb $hs$ liegen und
also auch von $s$ verschieden sind. Daraus folgt $M' \TeilVon as$, und da
offenbar $M'$ nicht Teil von $M$ sein kann (weil $M$ nicht $= S$ ist), so
muss es in $M$ ein Element $m$ geben, dessen Bild $m'$ außerhalb $M$, also
in $hs$ liegt, woraus $m's \TeilVon hs$ folgt.
\end{hilfssatz}

\begin{quote}
[\emph{Der Beweis ist offenbar unvollständig. Ein Beweis des Hilfssatzes ergibt
sich nach Mitteilung von J. Cavaillès direkt aus \sref{gthm25}, indem man die
dortigen $a$, $b$, $c$ durch $a$, $k$, $s$ ersetzt.
Der Zusatz folgt aus \sref{gthm28} und dem Hilfssatz. E. N.
}] \end{quote}

\begin{satz}\label{gthm29}
Ist $T$ ein Teil von $S$, und $s$ ein Element von $S$, so gibt es in $S$ immer ein
und nur ein zugehöriges Element $s_1$, welches die beiden folgenden Eigenschaften hat:
\begin{enumerate} \setcounter{enumi}{0} \setlength\itemsep{-0.25em}
   \item  Wenn $a$ der Bedingung $T \TeilVon as$ genügt, so ist $s_1s \TeilVon as$.
   \item $T \TeilVon s_1s$.
\end{enumerate}
Hieraus folgen die beiden Eigenschaften:
\begin{enumerate} \setcounter{enumi}{2} \setlength\itemsep{-0.25em}
    \item $s_1$ liegt in $T$.
    \item Die Strecke $ss_1$ enthält kein von $s$ und $s_1$ verschiedenes Element von $T$.
\end{enumerate}
\beweis
Da $s's = S$, also $T \TeilVon s's$ ist (\sref{gthm15}), so gibt es mindestens
ein Element $a$, das der Bedingung $T$ as genügt. Ist $A$ das System aller dieser
Elemente $a$, so ist (nach \sref{gthm28}) der Durchschnitt aller ihnen entsprechenden
Strecken eine Strecke $s_1s$, wo $s_1$ ein völlig bestimmtes Element von $S$ ist.
Nach dem Begriffe eines Durchschnitts hat $s_1$ die Eigenschaft 1.\, aber auch die
Eigenschaft 2.\, weil $T$ ein gemeinsamer Teil aller $as$, mithin auch Teil ihres
Durchschnitts $s_1s$ ist. Ist $s_1 = s$, also $s_1s = ss = [s]$, so folgt aus 2.\,
dass $T$ aus dem einzigen Elemente $s$ besteht; und umgekehrt, wenn $s$ in $T$
liegt und das einzige Element von $T$ ist, so ist $T = [s] = ss]$, also nach 1.\
auch $s_1s \TeilVon ss$, mithin $s_1 = s$; in diesem Falle hat daher $s_1$ die
Eigenschaft 3.\ und offenbar auch die Eigenschaft 4. Ist aber $s_1$ verschieden
von $s$, so ist (nach \sref{gthm18}) $s_1s = [s_1] + (s_1)'s$.

Nimmt man nun an, $s_1$ liege außerhalb $T$, es sei also jedes Element von $T$
verschieden von $s_1$, so folgt aus 2.\ auch $T \TeilVon (s_1)'s$, und hieraus
nach 1.\ auch $ss_1 \TeilVon (s_1)'s$, was aber unmöglich ist, weil das (nach
\sref{gthm10}) in $s_1s$ liegende Element $s_1$ (nach \sref{gthm19}) außerhalb
$(s_1)'s$ liegt. Mithin ist unsere Annahme unzulässig, d. h.  $s_1$  hat die
Eigenschaft 3.

Wir betrachten nun die Strecke $ss_1$; besitzt sie ein von $s$ und $s_1$
verschiedenes Element $u$, so ist auch $s$ verschieden von $s_1$ (weil sonst
$ss_1 = [s]$, also auch $u = s$ wäre), und (nach \sref{gthm18}) $ss_1 = [s] + ss_1$;
mithin liegt $u$ in $s's_1$, also (nach \sref{gthm19}) außerhalb $(s_1)'s$, und
da (wie oben) $s_1s = [s_1] + (s_1)'s$, und $u$ auch von $s_1$ verschieden ist, so
liegt $u$ auch außerhalb  $s_1s$, also zufolge 2. auch außerhalb $T'$, d. h. $s_1$
hat auch die Eigenschaft 4.
\end{satz}

\noindent \textbf{30.}\label{gthm30} \emph{Abbildung von $S$ nach $T$.}
Durch \sref{gthm29} ist eine Abbildung $\psi$ von $S$ in $T$ hergestellt, welche
dadurch definiert wird, dass jedes Element $s$ von $S$ durch $\psi$ in das dort
erklärte, (nach \sref{gdef3}) in $T$ liegende Element $s_1$ übergeht. Ist dann
$A$ irgend ein Teil von $S$, so soll $A_1$ das zugehörige Bild von $A$ (d. h.
das System der Bilder aller Elemente $a_1$ von $A$) bedeuten. Es ist also
$S_1 \TeilVon T$, also auch $T_1 \TeilVon T$, d. h. $T$ wird durch $\psi$ in sich
selbst abgebildet.

\stepcounter{satz}
\begin{satz}\label{gthm31}
Diese Abbildung $\psi$ von $T$ in sich selbst ist eine ähnliche, d. h. sind $a$, $b$
verschiedene Elemente von $T$, so sind auch deren Bilder $a_1$, $b_1$ verschieden.
\Beweis
Nach \sref{gthm29} ist $T \TeilVon a_1a$ und $T \TeilVon b_1b$. Da nun $a$, $b$
Elemente von $T$ sind, so ist auch $[a] \TeilVon b_1b$, $[b] \TeilVon a_1a$. Wäre
nun, obgleich $a$, $b$ verschieden sind, doch $a_1 = b_1 = c$, so wäre
$[a] \TeilVon cb$, $[b] \TeilVon ca$. Da aber $c$ von $a$ und $b$ verschieden
ist (weil sonst auch $a = b$ wäre), so ist dies (nach Zusatz zu \sref{gthm25})
unmöglich. Mithin sind $a_1$, $b_1$ verschieden, w.z.b.w.
\end{satz}

\newpage

\hspace{64pt}

\begin{center}\Large\textbf{Erläuterungen zur vorstehenden Abhandlung}
\end{center}

\bigskip

Die hier gegebene Definition des Endlichen ist chronologisch die erste, die die
Ableitung aller Eigenschaften ohne Heranziehung des Auswahlaxioms ermöglicht --
eine Tatsache, die Dedekind wohl noch nicht bewusst war.

Er selbst zieht nur die ersten Folgerungen; auf diesem Weg lässt sich aus seinem
letzten Satz noch folgern, dass jede Untermenge einer endlichen Menge endlich ist,
und es lässt sich das Prinzip der vollständigen Induktion beweisen und damit zu
der ursprünglichen Dedekindschen Definition übergehen (vgl. eine demnächst, Fund.
Math.\ \textbf{19}, erscheinende Note von J. Cavaillès).

Einen vergleichenden Überblick über die verschiedenen Definitionen des Endlichen
gibt A. Tarski (\emph{Sur les ensembles finis}, Fund. Math.\ \textbf{6}), dessen
eigene Definition so lautet: Eine Menge heißt endlich, wenn in jedem System von
Untermengen mindestens eine im System minimale enthalten ist. Gleichbedeutend mit
dieser Minimalbedingung -- durch Übergang zur Komplementärmenge -- ist die
entsprechende Maximalbedingung; aus beiden folgen alle Eigenschaften der endlichen
Mengen ohne Heranziehung des Auswahlpostulats.

Dass die obige Definition von Dedekind mit der Minimalbedingung äquivalent ist,
folgert Tarski aus der auch bei Dedekind in einer anderen Fassung auftretenden
Relation: $ab' \TeilVon ab + [b']$. Insbesondere gelangt Tarski so von der obigen
zu der ursprünglichen Dedekindschen Definition, während der umgekehrte Übergang
das Auswahlpostulat erfordert. Dedekind glaubte -- im Vorwort zur 2. Auflage von
\textit{\glqq Was sind und was sollen die Zahlen?\grqq{}} -- dass der Nachweis
der Übereinstimmung der Definitionen die volle dort entwickelte Theorie erfordere.

Wie er sich den Übergang im einzelnen gedacht hatte, zeigt die folgende Stelle
aus einem Brief an H. Weber:
\begin{quote}
\glqq Die kürzeste Charakterisierung des Endlichen und Unendlichen ist, wie ich
glaube, diejenige, welche ich am 9. März 1889 gefunden und in dem Vorwort (S. XI)
zur zweiten Auflage (1893) der Schrift \textit{\glqq Was sind und was sollen die
Zahlen?\grqq{}} mitgeteilt habe. Ich spreche sie so aus: \glq Ein System $S$ heißt
endlich, wenn es eine Abbildung von $S$ in sich selbst gibt, durch welche kein
echter Teil von $S$ in sich selbst abgebildet wird; im entgegengesetzten Falle
heißt $S$ ein unendliches System\grq{}.

\hspace{12pt} Nimmt man aber an, dass man die natürliche Zahlenreihe und ihre
Gesetze schon vollständig kennt, und
\end{quote}

\newpage

\begin{quote}
ersetzt man im vorstehenden das Wort \glq heißt\grq{} durch das Wort \glq ist\grq{},
so verwandelt sich diese Definition in einen Satz, der sich so beweisen lässt:

\hspace{12pt} Es sei $\phi$ eine Abbildung eines Systems $S$ in sich selbst, durch
welche kein echter Teil von $S$ in sich selbst abgebildet wird. Das Bild eines
Elementes $a$ oder eines Teiles $A$ von $S$ bezeichne ich mit $a\phi$ oder $A\phi$
(viel natürlicher als $\phi(a)$ oder $\phi(A)$). Ist $a$ irgendein Element von $S$,
so sind auch alle Bilder Elemente von $S$, also ist auch das System $A$ aller dieser
Bilder ein Teil von $S$,
\[
	a\phi, a\phi^2= (a\phi)\phi \ldots, a\phi^{n+1} = (a\phi^n)\phi \ldots
\]
und da $A\phi$ das System aller Bilder also ein Teil von $A$ ist,
\[
	(a\phi) \phi = a\phi^2, (a\phi^2)\phi = a\phi^3,
\]
so wird $A$ durch $\phi$ in sich selbst abgebildet; und folglich ist $A = S$.
Mithin ist $a$ auch Element von $A$,  es gibt also eine kleinste natürliche Zahl
$n$, die der Bedingung%
\[
	a\phi^n = a
\]
genügt. Dann ist $S$ das System der $n$ Elemente
\[
	a\phi,a\phi^2,\ldots a\phi^n
\]
und diese sind voneinander verschieden. Denn zufolge der Definition von $n$ ist das
letzte Element verschieden von allen vorhergehenden. Wäre ferner $1 \leq r<s < n$%
\[
	a\phi^r = a\phi^s
\]
so wäre
\[
	(a\phi^r)\phi^{n-s} = (a\phi^s) \phi^{n-s}
\]
also
\[
	a\phi^{r+n-s} = a\phi^n = a
\]
obgleich $1 < r + n - s < n$. Dass endlich $S$ keine anderen als diese $n$ Elemente
enthält, folgt aus $a\phi^{m + n} = a\phi^m$. Also ist wirklich $S$ ein endliches
System (im üblichen Sinne), und zugleich ergibt sich, dass $\phi$ eine zyklische
Permutation der $n$ Elemente von $S$, also auch eine ähnliche (d. h. eindeutig
umkehrbare) Abbildung ist.

\hspace{12pt} Umgekehrt, besteht ein (im üblichen Sinne) endliches System $S$ aus
$n$ verschiedenen Elementen
\[
	a_1, a_2 \ldots a_{n-1}, a_n
\]
und definiert man eine Abbildung $\phi$ von $S$ durch
\[
    a_n\phi = a_1, a_r\phi = a_{r+1}
\]
für $1 \leq r < n$, so ist $S' = S$, also $\phi$ eine Abbildung von $S$ in sich
selbst, und man zeigt leicht, dass kein echter Teil von $S$ in sich selbst
abgebildet wird. Denn wenn ein Teil $A$ von $S$ durch $\phi$ in sich selbst
abgebildet wird und ein Element $a$ enthält, so muss $A$ auch alle Elemente
$a\phi, a\phi^2, a\phi^3, \ldots\ $, also alle Elemente von $S$ enthalten,
mithin $= S$ sein. W.z.b.w.\grqq{}
\end{quote}

\vspace{-20pt} \begin{flushright} \textbf{Noether.}\end{flushright}


%%%%%%%%%%%%%%%%%%%%%%%%%%%%%%%%%%%%%%%%%%%%%%%%%%%%%%%%%%%%%%%%%%%%%%%%%%%%%%%%


\switchcolumn
\selectlanguage{english}

% \marginpar{\textit{What are numbers, and what is their purpose?}}%
\noindent First published in the second edition (1893) of the text
\textit{``Was sind und was sollen die Zahlen?''} page XVII, in the form:%

\begin{quote}
A system $S$ is called finite if it can be mapped into itself in such a way that
no proper part of $S$ is mapped into itself; in the opposite case, $S$ is called
an infinite system.
\end{quote}

Pursuing this definition of a finite system $S$ \emph{without using the natural numbers}.

Let $\varphi$ be a mapping of $S$ into itself, which maps no proper part of $S$
into itself. Small Latin letters $a, b \ldots z$ always mean \emph{elements} of
$S$, capital Latin letters $A, B \ldots Z$ mean \emph{parts} of $S$. The images
of $a, A$ generated by $\varphi$ are respectively denoted by $a', A'$.
\ \\

That $A$ is \emph{part of} $B$ is expressed by $A \partof B$. The system consisting
of the elements $a, b, c, \ldots$ is denoted by $[a, b, c, \ldots]$.
\ \\

This gives
\begin{equation}\label{eq1}
				S' \partof S
\end{equation}
and % \marginpar{\textit{The definition of finiteness.}}
\begin{equation}\label{eq2}
		\text{from } A' \partof A \text{ it follows that } A = S.
\end{equation}

\begin{satz}\label{thm1}$S' = S$.
\Beweis
Every element of $S$ is an image of (at least) one element $r$ of $S$. Because
from \eqref{eq1} it follows $(S')' \partof S'$, hence by \eqref{eq2}, our proposition.
\end{satz}
Every system $[s]$ consisting of a single element $s$ is finite because it has
no proper part and is mapped into itself by the identity function. This case is
\emph{excluded} in the following; $S$ means a finite system that does \emph{not}
consist of a single element.

\begin{satz}\label{thm2}
Every element $s$ is different from its image $s'$, in symbols: $s \neq s'$.
\Beweis
Because if $s = s'$, then $[s]' = [s'] = [s] \partof [s]$, so according to
\eqref{eq2}, also $[s] = S$ in contradiction to our assumption about $S$.
\end{satz}

\newpage

\begin{deff}\label{def3}
If $s$ is a certain element of $S$, then $H_s$ shall denote any part of $S$ that
satisfies the following two conditions:
\vspace{8pt} %\ \\%

\begin{enumerate}[I.]
\item $s$ is element of $H_s$, so $[s] \partof H_s$, also
\[
	[s] + H_s = H_s.
\]
\item If $h$ is an element of $H_s$ different from $s$, then $h'$ is also an
element of $H_s$. So if $H \partof H_s$, but $s$ is \emph{not contained} in H,
then $H' \partof H_s$.
\end{enumerate}
\end{deff}

\begin{satz}\label{thm4}
$S$ and $[s]$ are special systems $H_s$, and $[s]$ is the \emph{intersection}
(the common) of all systems $H_s$ corresponding to the element $s$.
\Beweis Obvious. \end{satz}

\begin{satz}\label{thm5}
$H_s = S$ or $H_s$ is a \textit{proper} part of $S$, depending on whether
$s'$ lies in $H_s$ or not.
\Beweis
For if $s'$ lies in $H_s$, then it follows from (\sref{gdef3}{\color{ultrav}.II)}
that $H'_s \partof H_s$, therefore by \eqref{eq2} that $H_s = S$. Conversely,
if $H_s = S$, then $s'$ also lies in $H_s$.
\end{satz}

\begin{satz}\label{thm6}
If $H_s$ is a \emph{proper} part of $S$, then $s'$ is the only element of
$H'_s$ that lies outside $H_s$.
\Beweis
This is because every element $k$ of $H'_s$ is the image $h'$ of at least one
element $h$ in $H$. If $k=h'$ is different from $s'$, then $h$ is also different
from $s$, and consequently (by \sref{gdef3}{\color{ultrav}.II)} $k = h'$ lies in
$H_s$, while the element $s'$ of $H'_s$ by \sref{thm5} lies outside $H_s$.
\end{satz}

\begin{satz}\label{thm7}
Every system $H'_s$ is a system $H_{s'}$, that is (by definition 3.):
\begin{enumerate}[I'.]
	\item $s'$ is element of $H'_s$
	\item If $k$ is an element of $H'_s$ that is different from $s'$, then $k'$
    also lies in $H'_s$.
\end{enumerate}
\beweis
The first follows from the fact that $s$ lies in $H_s$, the second from the fact
that $k$ lies in $H_s$ by \sref{thm6}.
\end{satz}

\begin{satz}\label{thm8}
If $A$, $B$, $C$ \ldots\ are special systems $H_s$ corresponding to the same $s$,
then their intersection $H$ is also a system $H_s$.
\Beweis
Because according to \sref{gdef3}{\color{ultrav}.I} $s$ is a common element of
$A, B, C, \ldots$, thus also an element of $H$. If $h$ is an element of $H$ that
is different from $s$, then by \sref{gdef3}{\color{ultrav}.II}, the image $h'$
is an element of $A$, of $B$, of $C$, \ldots, and therefore also of $H$. $H$ thus
fulfills the two conditions I and II in definition \sref{def3} that are
characteristic of every $H_s$.
\end{satz}

\begin{deff}\label{def9}
If $a$, $b$ are certain elements of $S$, then the symbol $ab$ (\emph{section} $ab$)
should mean the intersection of all those systems $H_b$ which (such as $S$)
contain the element $a$.%
\end{deff}

\newpage
\begin{satz}\label{thm10}
$a$ is an element of $ab$, i.e., $[a] \partof ab$.
\Beweis
This is because $ab$ is the intersection of all systems $H_b$ in which $a$ lies.
(So a is the \emph{start} of ab.)
\end{satz}

\begin{satz}\label{thm11}
$ab$ is a system $H_b$, i.e. $[b] \partof ab$, and if $s$ is an element of $ab$
different from $b$, then $[s'] \partof ab$.
\Beweis
This follows from \sref{thm8}.
\end{satz}

So $b$ is an element (the \emph{end}) of $ab$. If $H \partof ab$ but $b$ is not
contained in $H$, then $H' \partof ab$.

\begin{satz}\label{thm12}
From $[a] \partof H_b$, follows from $ab \partof H_b$.
\Beweis
Immediate consequence of definition \sref{def9}.
\end{satz}

\begin{satz}\label{thm13}
$aa = [a]$.
\Beweis
This follows from \sref{thm4}, because $aa$ is the intersection of all $H_a$
that contain the element $a$ according to \sref{def3}.I.
\end{satz}

\begin{satz}\label{thm14}
If $b'$ is an element of $ab$, then $ab = S$.
\Beweis
This follows from \sref{thm11} and \sref{thm5}.
\end{satz}

\begin{satz}\label{thm15}
$b' b = S$.
\Beweis
This follows from \sref{thm14} and \sref{thm10}.
\end{satz}

\begin{satz}\label{thm16}
If $c$ is an element of $ab$, then $cb \partof ab$.
\Beweis
This follows from \sref{thm12}, since $ab$ is an $H_b$ by \sref{thm11}, that
contains the element $c$.
\end{satz}

\begin{satz}\label{thm17}
If $A + B$ means the system composed of $A, B$, then one has
\[
	a'b + b'a = S.
\]%
\beweis
Because if $s$ is an element of $a'b$, then $s'$ is contained in $b'a$ or $a'b$,
depending on $s = b$ or different from $b$ (according to \sref{thm10} or
\sref{thm11} and \sref{def3}.II), and likewise if $s$ is an element of $b'a$,
then $s'$ is contained in $a'b$ or $b'a$; therefore $(a'b + b'a)' \partof a'b + b'a$.
This leads to the theorem according to \eqref{eq2}.
\end{satz}

\begin{satz}\label{thm18}
If $a$ is different from $b$, then $ab = [a] + a'b$.
\Beweis
For since $a$ is an element of $ab$ different from $b$, then $a'$ is an element
of $ab$ (by \sref{thm10}, \sref{thm11}), and consequently (by \sref{thm16})
$a'b \partof ab$; since furthermore, by \sref{thm10}, we also have $[a]
\partof ab$, therefore
\[
	[a] + a'b \partof ab.
\]
Also, every element $s$ of $[a] + a'b$ that is different from $b$ is either
$= a$ or an element of $a'b$ that is different from $b$. Thus in both cases $s'$
is (by \sref{thm10}, \sref{thm11}) an element of $a'b$, therefore also of
$[a]+ a'b$, and since by \sref{thm11}) also $[b] \partof [a] + a'b$, it follows
that $[a] + a'b$ is a system $H_b$. Finally, since $[a] \partof [a] + a'b$, by
\sref{thm12} also
\[
	ab \partof [a] + a' b.
\]
The theorem follows from the comparison of both results.
\end{satz}

\newpage

\begin{satz}\label{thm19}
If $a, b$ are different elements of $S$, then $a$ lies outside $a'b$, and $b$ lies
outside $b'a$.
\Beweis
If one assumes the opposite, that there is an element $a$ that is different from
$b$ and lies in $a'b$, and that $A$ denotes the system of all such elements $a$,
the following holds.

If one puts $a'=s$, then $a$ lies in $sb$, and since $a$ is different from $b$,
and therefore (according to \sref{thm13}) is not in $bb$, then $s$ is different
from $b$, and from this it follows (according to \sref{thm18}) that $sb = [s] + s'b$.
Furthermore, since $a$ (according to \sref{thm2}) is different from $s$ and lies
in $sb$, then $a$ must lie in $s'b$, and from this it follows (again according to
\sref{thm1}) that $s$ (as the image $a'$) also lies in $s'b$.%
Therefore, the image $a'$ of every element $a$ of $A$ is also contained in $A$,
i.e. $A' \partof A$. But since $A=S$ would follow from this, while $A$ does not
contain the element $b$, our assumption is inadmissible, so the theorem is true,
qed.
The second part follows by exchanging $a$ with $b$.%
\end{satz}

\begin{satz}\label{thm20}
If $a$, $b$ are different, then the segments $a'b$, $b'a$ have no common element.
\Beweis
If one assumes the opposite, that there is a common element $m$ of $a'b$, $b'a$,
then it follows from the preceding Theorem \sref{thm19} that $m$ is different from
$b$ and from $a$; therefore (according to \sref{thm11}) the image $m'$ must also be
a common element of $a'b$ and $b'a$.%

Therefore, if M denotes the system of all such elements $m$, then $M' \partof M$,
thus $M=S$. But this is impossible because $a$, $b$ are elements of $S$ but not
elements of $M$. So our theorem is true.%
\ \\
\end{satz}

\begin{satz}\label{thm21}
If $a$, $b$ are different, then the images $a'$, $b'$ are also different.
\Beweis
Otherwise the segments $a'b$, $b'a$ would have a common element $a'=b'$, because
(according to \sref{thm10}) $a'$ is an element of $a'b$ and $b'$ is an element
of $b'a$.%
\end{satz}

\begin{satz}\label{thm22}
From $cb=S$ follows $c = b$.
\Beweis
There is (according to \sref{thm1} and \sref{thm21}) in $S$ one and only one element
$a$ which satisfies the condition $a'=c$, and therefore $a'b = S$, therefore
$[a] \partof a'b$; therefore (by \sref{thm19}) $a=b$, thus $c=b'$, qed.%
\end{satz}

\begin{satz}\label{thm23}
If $a$, $b$ are different, then every element of $S$ is contained in one and only
one of the segments $a'b$, $b'a$.
\Beweis
This follows from \sref{thm17} and \sref{thm20}.
\end{satz}

\begin{satz}\label{thm24}
If $a$, $b$, $c$ are different, then the segments $b'c$, $c'a$, $a'b$ have no
common element, and the same applies to the segments $a'c$, $b'a$, $c'b$.
\Beweis
Because the opposite assumption, that there is an element $m$ common to the segments
$b'c$, $c'a$, $a'b$, leads to a contradiction. Let $M$ be the system of all such
elements. Since (according to \sref{thm19}) $a$ is not in $a'b$, $b$ is not in $b'c$,
$c$ is not in $c'a$, then $m$ is different from $c$, $a$, $b$, and consequently
(by \sref{thm11}) $m'$ is a common element of $b'c$, $c'a$, $a'b$, i.e. an element
of $M$; therefore $M' \partof M$, hence  $M=S$.

But this is impossible because $M$ does not contain any of the elements $a$, $b$,
$c$. So our theorem is true.

The second part results from the first if one swaps $a$ with $b$, which does not
change the assumption.
\end{satz}

\begin{zusatz}\label{corollary1}
If you put (as in the following \sref{thm25}):
\[
    A = c'b,\ B = a'c,\ C = b'a;\  A_1 = b'c,\ B_1 = c'a,\ C_1 = a'b,
\]
then $A - B - C = 0$ (empty) (the symbol $-$ denotes intersection) and
$A_1 - B_1 - C_1 = 0$ (empty) and (according to \sref{thm17}, \sref{thm20}) hence
\begin{gather*}
    S = A + A = B + B_1 = C + C_1; \\
    \,0 = A - A_1 = B - B_1 = C - C_1.
\end{gather*}
This also applies (according to \sref{thm20}) if at least two of the elements
$a$, $b$, $c$ are different.
\end{zusatz}

\begin{satz}\label{thm25}
If $a$, $b$, $c$ are different, then one and only one of the following two
cases occurs: Either
\begin{gather*}
    b'c = b'a + a'c, \quad c'a = c'b + b'a, \quad a'b = a'c + c'b \\
    c'b = c'a - a'b, \quad a'c = a'b - b'c, \quad b'a = b'c - c'a
\end{gather*}
and each element of S lies in one, but only one, of the segments
$c'b$, $a'c$, $b'a$; or
\begin{gather*}
    c'b = c'a + a'b, \quad a'c = a'b + b'c, \quad b'a = b'c + c'a \\
    b'c = b'a - a'c, \quad c'a = c'b - b'a, \quad a'b = a'c - c'b
\end{gather*}
and each element of $S$ lies in one, but only one, of the segments
$b'c$, $c'a$, $a'b$.
\Beweis
According to \sref{thm23}, $c$ lies either in $a'b$ or in $b'a$. We
only consider the first case because the second arises from it by
exchanging $a$ for $b$.
Since $c$ is in $a'b$ and is distinct from $b$, then
(according to \sref{thm11}) $c'$ also lies in $a'b$, and consequently
(by \sref{thm16}) $c'b \partof a'b$; from this it follows (by \sref{thm19})
that $c'b$ has no element in common with $b'a$; now (by \sref{thm17}) is
$a'b+b'a=b'c+c'b$, therefore $b'a \partof b'c$, and consequently
(by \sref{thm11}) $a$ is in $b'c$.

From the assumption that $c$ lies in $a'b$, it follows: $c'b \partof a'b$,
$b'a \partof b'c$, $a$ lies in $b'c$. In the same way, this last
conclusion follows if one assumes $c$, $a$, $b$ replaced by $a$, $b$, $c$,
respectively, again we have the consequences $a'c \partof bc$, $cb \partof c'a$,
and that $b$ lies in $c'a$; and from this it follows again $b'a \partof c'a$,
$a'c \partof a'b$ (and the first assumption: $c$ lies in $a'b$).

Therefore: $c'b \partof a'b$, $b'a \partof b'c$, $a'c \partof b'c$,
$c'b \partof ca$, $b'a \partof c'a$, $a'c \partof a'b$, and also
$b'a + a'c \partof b'c$, $c'b + b'a \partof c'a$, $a'c + c'b \partof a'b$.

\newpage

If now an element of, say, $b'c$ is neither in $b'a$ nor in $a'c$, then
(by \sref{thm23}) it would be a common element of $b'c$, $a'b$, $c'a$,
which (by \sref{thm24}) is impossible; therefore $b'c \partof b'a + a'c$,
therefore also $b'c=b'a + a'c$, and likewise follows $c'a=c'b+b'a$, $a'b=a'c+c'b$.

If, say, $b'a$, $a'c$ have a common element, then the same would also be a common
element of $b'c$, $c'a$, $a'b$, which (according to \sref{thm24}) is not the case.
From $S=b'c+c'b$ we finally get $S=b'a+a'c+c'b$, which means our theorem is
completely proven.
\end{satz}

\begin{zusatz}
It can never be that $[a] \partof cb$ and $[b] \partof ca$ at the same time;
because (according to \sref{thm18}) then $[a] \partof c'b$ and $[b] \partof c'a$
would have to be at the same time, which is impossible.
\end{zusatz}

\begin{satz}\label{thm26}
From $ab=cb$ it follows that $a=c$, and if $ab=cd$ is a proper part of $S$,
then $a=c$, $b = d$.
\Beweis
This follows from earlier theorems.
Since (by \sref{thm10}) $c$ is in $cb$, and therefore also in $ab$, then if
$a=b$, then $ab = [a]$, then must also $c=a$. But if $a$ is different from $b$,
then (by \sref{thm18}) $ab = [a]+a'b$, and (by \sref{thm19}) $a'b$ is a proper
part of $ab$. If one assumes that $c$ is different from $a$, then $c$ must lie
in $a'b$, so (by \sref{thm16}) $cb \partof a'b$, i.e. $cb$ is a proper part of $ab$;
But since $cb=ab$, this assumption is inadmissible, and therefore always $c=a$, qed.

Furthermore, if $ab=cd$ is a proper part of $S$, then $b=d$; If $b$ is different
from $d$, then (by \sref{thm11}) $b'$ must also be in $cd$, and therefore also in
$ab$; But then (by \sref{thm14}) would $ab = S$ violating the assumption, so $b=d$,
therefore $ab=cb$, therefore also $a=c$, qed.
\end{satz}

\begin{satz}\label{thm27}
Every system $H_s$ (defined in \sref{def3}) is a segment $a's$ with the end $s$
and its beginning $a'$ completely determined.
\Beweis
If $H_s= S$, then $H_s= s's$ (according to \sref{thm15}).
But if $H_s$ is a proper part of $S$, then $A$ is the system of all elements
of $S$ that lie outside $H_s$, so $S=A+H_s$. Since $A$ is a proper part of $S$,
then $A' \partof A$ cannot be, so there is certainly an element $a$ in $A$, whose
image $a'$ lies outside $A$, hence in $H_s$. Since (according to \sref{thm12})
$a's \partof H_s$, then $a's$ and $A$ have no common element.

Since $a$ is in $A$, $s$ is in $H_s$ (even in $a's$), then $a$ and $s$ are
different, so (by \sref{thm20}) the segments $a's$, $s'a$ have no common element,
and (by \sref{thm17}) $a's+  s'a=S=H_s+ A,$ therefore $A \partof s'a$.
If one now assumes that $a's$ is a proper part of $H_s$, and $H$ denotes the
system of all those elements of $H_s$ which are outside $a's$, i.e., in $s'a$,
then $H_s = H+a's$, and $s'a = H + A$, so $H=H_s-s'a$ is the intersection of the
systems $H_s$, $s'a$.

Since neither $s$ nor $a$ lies in $H$, it follows from $H \partof H_s$, and
$H \partof s'a$ (according to \sref{def3} and \sref{thm11}) that $H' \partof H_s$
and $H' \partof s'a$, therefore also $H' \partof H$, hence $H = S$. But this is
impossible because $s$ (and also $a$) lies outside $H$. Therefore certainly
$H_s = a's$, and $A = s'a$, qed.
\end{satz}

\newpage

\begin{satz}\label{thm28}
The intersection of such segments $as, bs, \ldots$ which have the same end $s$ is
itself such a segment $hs$, and its beginning $h$ is completely determined.
\Beweis
For every such segment is (according to \sref{thm11}) a system $H_s$, and
(according to \sref{thm8}) the same applies to its intersection, from which the
theorem (according to \sref{thm27}) follows.
\end{satz}

\begin{zusatz} \label{cor_to_thm28}
The intersection of the segments $as, bs, cs \ldots$ is itself one of these segments.
\end{zusatz}

To prove it, we first provide the following lemma:

\begin{hilfssatz}
If $hs$ is a proper part of $as$, and $k$ is the element whose image is $k'=h$,
then $hs$ is also a proper part of $ks$, and at the same time $ks \partof as$.
\Beweis
If $k=s$, then $hs=s's= S$, while $hs$ is a proper part of $as$, and therefore
also of $S$. Since $k$ is different from $s$, then (by \sref{thm18}) $ks = [k]+ hs$,
and (according to \sref{thm19}) $k$ is not contained in $hs$, so $hs$ is a proper
part of $ks$. Since $hs$ is a proper part of $as$, let $as = M + hs$, where $M$
is the system of all elements $m$ of $as$ that lie outside $hs$ and are therefore
also different from $s$. From this follows $M' \partof as$, and since $M'$ obviously
cannot be part of $M$ (because $M$ is not $= S$), there must be in $M$ an element
$m$, the image $m'$ of which lies outside $M$, hence in $hs$, from which
$m's \partof hs$ follows.
\end{hilfssatz}

\begin{quote}
[\emph{The proof is apparently incomplete. According to J. Cavaillès, a proof of
the Lemma results directly from \sref{thm25} by replacing the $a$, $b$, $c$ there
by $a$, $k$, $s$. The Corollary follows from \sref{thm28} and the Lemma.
\ \\
E. N.
}]\end{quote}


\begin{satz}\label{thm29}
If $T$ is a part of $S$, and $s$ is an element of $S$, then in $S$ there is always
one and only one associated element $s_1$, which has the following two properties:
\begin{enumerate} \setcounter{enumi}{0} \setlength\itemsep{-0.25em}
  \item If $a$ satisfies the condition $T \partof as$, then $s_1s \partof as$.
  \item $T \partof s_1s$.
\end{enumerate}
From this follow the two properties:
\begin{enumerate} \setcounter{enumi}{2} \setlength\itemsep{-0.25em}
  \item $s_1$ is in $T$.
  \item the segment $ss_1$ contains no element of $T$ that is different from $s$ and $s_1$.
\end{enumerate}
\beweis
Since $s's = S$, therefore $T \partof s's$ (by \sref{thm15}), there is at least
one element $a$ that satisfies the condition $T \partof as$. If $A$ is the system
of all such elements $a$, then (according to \sref{thm28}) the intersection of
all the segments corresponding to them is a segment $s_1s$, where $s_1$ is a
completely determined element of $S$. According to the concept of an intersection,
$s_1$ has the property 1., but also property 2., because $T$ is a common part of
all $as$, and therefore also part of their intersection $s_1s$. If $s_1 = s$,
then $s_1s = ss = [s]$, then it follows from 2.\ that $T$ consists of the single
element $s$; and vice versa, if $s$ lies in $T$ and is the only element of $T$,
then $T = [s] = ss$, so then according to 1. $s_1s \partof ss$, therefore $s_1 = s$.
In this case, $s_1$ therefore has the property 3.\ and obviously also the property
4.\ But if $s_1$ is different from $s$, then (according to \sref{thm18})
$s_1s = [s_1] + (s_1)'s$.

If one now assumes that $s_1$ lies outside $T$, if every element of $T$ is
different from $s_1$, it follows from 2.\ also that $T \partof (s_1)'s$, and from
this according to 1.\ we also have $s_1s \partof (s_1)'s$, but this is impossible
because (according to \sref{thm10}) in $s_1s$ is the element $s_1$ (according to
\sref{thm19}) which lies outside $(s_1)'s$; therefore our assumption is inadmissible,
i.e., $s_1$ has property 3.

We now consider the segment $ss_1$; if it has an element $u$ that is different
from $s$ and $s_1$, then $s$ is also different from $s_1$ (because otherwise
$ss_1 = [s]$, which would also give $u=s$), and (according to \sref{thm18})
$ss_1 = [s] + ss_1$; therefore $u$ lies in $s's_1$, therefore (according to
\sref{thm19}) outside $(s_1)'s$, and since (as above) $s_1s= [s_1] + (s_1)'s$,
and $u$ is also different from $s_1$, then $u$ also lies outside $s_1s$, therefore
according to 2.\ also outside $T'$, i.e., $s_1$ also has property 4.
\end{satz}

\noindent \textbf{30.}\label{thm30} \hspace{-10pt} \emph{Mapping of $S$ into $T$}.
Through \sref{thm29} a mapping $\psi$ of $S$ into $T$ is
created, which is defined by the fact that each element $s$ of $S$ is sent by
$\psi$ into the element $s_1$, which is defined there and (according to \sref{def3})
lies in $T$. If $A$ is then any part of $S$, then $A_1$ should mean the associated
image of $A$ (i.e., the system of images $a_1$ of all elements $a$ of $A$). So
$S_1 \partof T$, also $T_1 \partof T$, i.e. $T$ is mapped by $\psi$ into itself.

\stepcounter{satz}
\begin{satz}\label{thm31}
This mapping of $T$ into itself is a similar one, i.e., if $a, b$ are different
elements of $T$, then their images $a_1, b_1$ are also different.%
\Beweis
By \sref{thm29}, $T \partof a_1a$ and $T \partof b_1b$. Since $a, b$ are elements
of $T$, then $[a] \partof b_1b$, $[b] \partof a_1a$. If, although $a, b$ are
different, $a_1 = b_1  = c$, so then $[a] \partof cb, [b] \partof c$. But since
$c$ is different from $a$ and $b$ (because otherwise $a = b$), this is impossible
(after the Corollary to \sref{thm25}). Therefore $a_1, b_1$ are different, qed.
\end{satz}

\newpage

\hspace{84pt}

\begin{center}{\Large\textbf{Explanations to the above treatise \ \\ }}
\end{center}

\vspace{36pt}

The definition of the finite given here is chronologically the first that enables
the derivation of all properties without using the axiom of choice -- a fact that
Dedekind was probably not yet aware of.

He himself only draws the first conclusions. In this way, one can still conclude
from his last theorem that every subset of a finite set is finite, and the principle
of complete induction can be proven and thus move on to Dedekind's original
definition (cf. a forthcoming note by J. Cavaillès, Fund.\ Math.\ \textbf{19}.)
\ \\

A comparative overview of the various definitions of finiteness is given by A.
Tarski (\emph{Sur les ensembles finis}, Fund.\ Math.\ \textbf{6}) whose own
definition goes like this: A set is said to be finite if in every system of
subsets at least one is included that is minimal in the system. The corresponding
maximum condition is equivalent to this minimum condition -- through transition to the
complementary set; all properties of finite sets follow from both without using
the axiom of choice.
\ \\

Tarski concludes that the above definition by Dedekind is equivalent to the
minimal condition from the relation that also appears in a different version in
Dedekind: $ab' \partof ab+[b']$. In particular, Tarski gets from the above to
the original Dedekind definition, while the reverse transition requires the axiom
of choice. Dedekind believed -- in the preface to the 2nd edition of
\emph{``Was sind und was sollen die Zahlen?''} -- that the proof of the agreement
of the definitions required the full theory developed there.
\ \\

How he thought about the transition in detail is shown in the following passage
from a letter to H. Weber:

\begin{quote}
``The shortest characterization of the finite and infinite is, as I believe, the
one which I found on March 9, 1889 and in the preface (p. XI) to the second edition
(1893) of the work, \emph{``Was sind und was sollen die Zahlen?''}. I say it like
this: `A system $S$ is called finite if there is a mapping of $S$ into itself
through which no proper part of $S$ is mapped into itself; in the opposite case,
$S$ is called an infinite system'.
\ \\

\hspace{12pt} But if one assumes that one already knows the series of natural
number and its laws completely, and

\end{quote}

\newpage

\begin{quote}
one replaces the word `called' with the word `is' in the above, then this definition
turns into a theorem that can be proven like this:

\hspace{12pt} Let $\phi$ be a mapping of a system $S$ into itself, through which
no proper part of $S$ is mapped into itself. I denote the image of an element $a$
or a part $A$ of $S$ with $a\phi$ or $A\phi$ (much more natural than $\phi(a)$
or $\phi(A)$). If $a$ is any element of $S$, then all the images are elements of
$S$, thus the system $A$ of all these images is also a part of $S$,
\vspace{16pt}
\[
	a\phi, a\phi^2= (a\phi)\phi \ldots, a\phi^{n+1} = (a\phi^n)\phi \ldots
\]
and since $A\phi$ is the system of all images is also a part of $A$,
\[
	(a\phi) \phi = a\phi^2, (a\phi^2)\phi = a\phi^3,
\]
then $A$ is mapped into itself by $\phi$; and consequently $A = S$. Therefore $a$
is also an element of $A$, so there is a smallest natural number $n$, that
satisfies the condition
\[
	a\phi^n = a.
\]
Then $S$ is the system of $n$ elements
\[
	a\phi,a\phi^2,\ldots a\phi^n
\]
and these are different from each other.
For according to the definition of $n$, the last element is different from all
previous ones. Assuming further $1 \leq r<s < n$ and
\[
	a\phi^r = a\phi^s
\]
than it would be the case that
\[
	(a\phi^r)\phi^{n-s} = (a\phi^s) \phi^{n-s}
\]
hence
\[
	a\phi^{r+n-s} = a\phi^n=a
\]
although $1<r+n-s<n$.
Finally, the fact that $S$ contains no elements other than these follows from
$a\phi^{m + n} = a\phi^m$. So $S$ is really a finite system (in the usual sense),
and at the same time it follows that $\phi$ is a cyclic permutation of the $n$
elements of $S$, so it is also a similar (i.e. clearly revertible) mapping.

\hspace{12pt} Conversely, a finite (in the usual sense) system $S$ consists of
$n$ different elements
\[
	a_1, a_2 \ldots a_{n-1}, a_n
\]
and one defines a mapping $\phi$ of $S$ by
\[
a_n\phi = a_1, a_r\phi=a_{r+1}
\]
for $1\leq r<n$, then $S' = S$, so $\phi$ is a mapping from $S$ into itself, and
one easily shows that no proper part of $S$ is mapped into itself. Because if
a part $A$ of $S$ is mapped by $\phi$ into itself and contains an element $a$,
then $A$ must also contain all elements $a\phi, a\phi^2, a\phi^3, \ldots \ $, i.e.,
all elements of $S$, and therefore be $= S$. Qed.''
\end{quote}

\begin{flushright}\textbf{Noether.}\end{flushright}

\end{paracol}

\end{document}
