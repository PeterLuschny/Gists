\documentclass[leqno,hidelinks]{article}
\usepackage[german,english]{babel}
\usepackage{amsmath,amssymb,amsthm}
\usepackage{color,hyperref}
\usepackage{enumerate}
\usepackage{microtype}
\linespread{1.25}

\definecolor{darkblue}{rgb}{0,0,.7}
\definecolor{ultrav}{rgb}{0.5,0,1}

\hypersetup{colorlinks,
            breaklinks,
            linkcolor=ultrav,
            urlcolor=darkblue,
            anchorcolor=darkblue,
            citecolor=darkblue}

\newtheoremstyle{}
        {}%space above
        {}%space below
        {}%body font
        {}%Indent amount
        {\bfseries}%Theorem head font
        {}%Punctuation after theorem head
        {.5em}%Space after theorem head
        {#1.#2.~\thmnote{#3.}
}%Theorem head spec
\swapnumbers

\theoremstyle{definition}

\newtheorem{satz}{\protect\satzname}
\newtheorem{deff}[satz]{\protect\deffname}
\newtheorem*{zusatz}{\protect\zusatzname}
\newtheorem*{hilfssatz}{\protect\hilfssatzname}

\newcommand{\satzname}{}
\newcommand{\deffname}{}
\newcommand{\zusatzname}{}
\newcommand{\hilfssatzname}{}

\addto\captionsgerman{%
    \renewcommand{\satzname}{\hspace{-4pt}.\ Satz}%
    \renewcommand{\deffname}{\hspace{-4pt}.\ Definition}%
    \renewcommand{\zusatzname}{Zusatz}%
    \renewcommand{\hilfssatzname}{Hilfssatz}%
}

\addto\captionsenglish{%
    \renewcommand{\satzname}{\hspace{-4pt}.\ Theorem}%
    \renewcommand{\deffname}{\hspace{-4pt}.\ Definition}%
    \renewcommand{\zusatzname}{Corollary}%
    \renewcommand{\hilfssatzname}{Lemma}%
}

\newcommand\Beweis{\medskip \newline $ \phantom{'.'} \rhd \ $}%
\newcommand\beweis{ $ \phantom{'.'} \rhd \ $}%

% From https://narkive.com/P4sNM0k5:7.443.63
\newcommand\TeilVon{\mathrel{\raisebox{0.45ex}{$\mathfrak{3}$}}}
\newcommand{\partof}{\in}

%\pagestyle{myheadings}
%\markright{Dedekind, Second definition of the finite and infinite\hfill}

\usepackage{paracol}
%\usepackage[landscape, total={8in, 8in}]{geometry}
\usepackage[total={8in, 10in}]{geometry}

% \usepackage{showframe}
% \setcounter{page}{451}

\begin{document}
\hypersetup{pageanchor=false}

\title{Zweite Definition des Endlichen und Unendlichen.}
\author{Richard Dedekind}
\date{1889. 3. 9. \phantom{} [ 9th March 1889 ]}
\maketitle

\thispagestyle{empty}

\begin{paracol}{2} % specify two columns...

\selectlanguage{german}

\noindent Zuerst veröffentlicht in der zweiten Auflage (1893) der Schrift
\textit{\glqq Was sind und was sollen die Zahlen?\grqq{}} Seite XVII, in der Form:%

\begin{quote}
Ein System $S$ heißt endlich, wenn es sich so in sich selbst abbilden
lässt, dass kein echter Teil von $S$ in sich selbst abgebildet
wird; im entgegengesetzten Fall heißt $S$ ein unendliches System.
\end{quote}

Verfolgung dieser Definition eines endlichen Systems $S$
\emph{ohne Benutzung der natürlichen Zahlen}.

Es sei $\varphi$ eine Abbildung von $S$ in sich selbst, durch welche kein echter
Teil von $S$ in sich selbst abgebildet wird.
Kleine lateinische Buchstaben  $a, b \ldots z$ bedeuten immer \emph{Elemente} von
$S$, große lateinische Buchstaben $A, B \ldots Z$ bedeuten \emph{Teile} von $S$;
die durch $\varphi$ erzeugten Bilder von $a, A$ werden resp. mit $a', A'$ bezeichnet.

Dass $A$ \emph{Teil von} $B$ ist, wird durch $A \TeilVon B$ ausgedrückt. Das aus den
Elementen $a, b, c, \ldots $ bestehende System wird mit $[a, b, c \ldots]$ bezeichnet.

Es ist also
\begin{equation}\label{geq1}
				S' \TeilVon S
\end{equation}
und% \marginpar{\textit{Die Definition des Endlichen.}}
\begin{equation}\label{geq2}
		\text{aus } A' \TeilVon A \text{ folgt } A = S.
\end{equation}

\begin{satz}\label{gthm1}$S' = S$.
\Beweis
Jedes Element von $S$ ist Bild von (mindestens) einem Element $r$ von $S$. Denn
aus \eqref{geq1} folgt $(S')' \TeilVon S'$, also nach \eqref{geq2} unser Satz.
\end{satz}
Jedes aus einem einzigen Element $s$ bestehende System $[s]$ ist endlich, weil
es keinen echten Teil besitzt und durch die identische Abbildung in sich selbst
abgebildet wird. Dieser Fall wird im folgenden \emph{ausgeschlossen}, $S$ bedeutet
ein endliches System, das \emph{nicht} aus einem einzigen Element besteht.

\smallskip

\begin{satz}\label{gthm2}
Jedes Element $s$ ist verschieden von seinem Bilde $s'$, in Zeichen: $s \neq s'$.
\Beweis
Denn wäre $s = s'$, so wäre $[s]' = [s'] = [s] \TeilVon [s]$, nach \eqref{geq2}
auch $[s] = S$ im Widerspruch zu unserer Annahme über $S$.
\end{satz}

\newpage

\begin{deff}\label{gdef3}
Ist $s$ ein bestimmtes Element von $S$ so soll mit $H_s$ jeder solche Teil von
$S$ bezeichnet werden, der den beiden folgenden Bedingungen genügt:
\begin{enumerate}[I.]
\item $s$ ist Element von $H_s$, also $[s] \TeilVon H_s$, also auch
\[
	[s] + H_s = H_s.
\]
\item Ist $h$ ein von $s$ verschiedenes Element von $H_s$, so ist auch $h'$
Element von $H_s$; ist also $H \TeilVon H_s$, aber $s$ nicht in $H$ enthalten,
so ist $H' \TeilVon H_s$.
\end{enumerate}
\end{deff}

\begin{satz}\label{gthm4}
$S$ und $[s]$ sind spezielle Systeme $H_s$, und $[s]$ ist der Durchschnitt
(die Gemeinheit) aller dem Elemente $s$ entsprechenden Systeme $H_s$.
\Beweis Offenbar.
\end{satz}

\begin{satz}\label{gthm5}
$H_s = S$ oder \textit{echter} Teil von $S$, je nachdem $s'$ in $H_s$ liegt oder nicht.
\Beweis
Denn wenn $s'$ in $H_s$ liegt, so folgt aus (\ref{gdef3}.II), dass $H'_s \TeilVon H_s$,
also nach \eqref{geq2}, dass $H_s = S$ ist; und umgekehrt, wenn $H_s = S$, so liegt
auch $s'$ in $H_s$.
\end{satz}

\begin{satz}\label{gthm6}
Ist $H_s$ \emph{echter} Teil von $S$, so ist $s'$ das einzige Element von $H'_s$,
das außerhalb $H_s$ liegt.
\Beweis
Denn jedes Element $k$ von $H'_s$ ist Bild $h'$ von mindestens einem Element $h$
in $H_s$; ist nun $k=h'$ verschieden von $s'$, so ist auch $h$ verschieden von $s$,
und folglich nach (\ref{gdef3}.II) liegt $k = h'$ in $H_s$, während das Element $s'$
von $H'_s$ nach (\ref{gthm5}) außerhalb $H_s$ liegt.
\end{satz}

\begin{satz}\label{gthm7}
Jedes System $H'_s$ ist ein System $H_{s'}$, das heißt (Definition (\ref{gdef3})):
\begin{enumerate}[I'.]
    \item $s'$ ist Element von $H'_s$.
    \item Ist $k$ ein von $s'$ verschiedenes Element von $H'_s$, so liegt auch $k'$ in $H'_s$.
\end{enumerate}
\beweis
Das Erste folgt daraus, dass $s$ in $H_s$ liegt, das Zweite daraus,
dass nach Satz (\ref{gthm6}) $k$ in $H_s$ liegt.
\end{satz}

\begin{satz}\label{gthm8}
Sind $A$, $B$, $C$ \ldots\ spezielle, demselben $s$ entsprechende Systeme $H_s$,
so ist auch ihr Durchschnitt $H$ ein System $H_s$.
\Beweis
Denn zufolge (\ref{gdef3}.I) ist $s$ gemeinsames Element von $A, B, C, \ldots$,
also auch Element von $H$. Ist ferner $h$ ein von $s$ verschiedenes Element von
$H$, so ist zufolge (\ref{gdef3}.II) das Bild $h'$ Element von $A$, von $B$, von
$C$, \ldots, also auch von $H$. Mithin erfüllt $H$ die beiden für jedes $H_s$
charakteristischen Bedingungen I, II in (\ref{gdef3}).
\end{satz}

\begin{deff}\label{gdef9}
Sind $a$, $b$ bestimmte Elemente von $S$, so soll das Symbol $ab$ den Durchschnitt
aller derjenigen Systeme $H_b$ bedeuten (\emph{Strecke} $ab$), welche (wie z. B. $S$)
das Element $a$ enthalten.
\end{deff}

\newpage

\begin{satz}\label{gthm10}
$a$ ist Element von $ab$, d. h. $[a] \TeilVon ab$.
\Beweis
Denn $ab$ ist der Durchschnitt von lauter solchen Systemen $H_b$ in denen $a$
liegt. (a \emph{Anfang} von $ab$.)
\end{satz}

\begin{satz}\label{gthm11}
$ab$ ist ein System $H_b$, d. h. $[b] \TeilVon ab$, und wenn $s$ ein von $b$
verschiedenes Element von $ab$ ist, so ist $[s'] \TeilVon ab$.
\Beweis
Dies folgt aus (\ref{gthm8}).
\end{satz}

Also $b$ Element (\emph{Ende}) von $ab$. Ist $H \TeilVon ab$, aber $b$ nicht in
$H$ enthalten, so ist $H' \TeilVon ab$.

\begin{satz}\label{gthm12}
Aus $[a] \TeilVon H_b$ folgt $ab \TeilVon H_b$.
\Beweis
Unmittelbare Folge von (\ref{gdef9}).
\end{satz}

\begin{satz}\label{gthm13}
$aa = [a]$.
\Beweis
Dies folgt aus (\ref{gthm4}), weil $aa$ der Durchschnitt aller $H_a$  ist, die
ja alle das Element $a$ enthalten nach (\ref{gdef3}.I).
\end{satz}

\begin{satz}\label{gthm14}
Ist $b'$ Element von $ab$, so ist $ab = S$.
\Beweis
Dies folgt aus (\ref{gthm11}) und (\ref{gthm5}).
\end{satz}

\begin{satz}\label{gthm15}
$b' b = S$.
\Beweis
Dies folgt aus (\ref{gthm14}) und (\ref{gthm10}).
\end{satz}

\begin{satz}\label{gthm16}
Ist $c$ Element von $ab$, so ist $cb \TeilVon ab$.
\Beweis
Dies folgt aus (\ref{gthm12}), denn $ab$ ist ein $H_b$,
(nach (\ref{gthm11})), welches das Element $c$ enthält.
\end{satz}

\begin{satz}\label{gthm17}
Bedeutet $A + B$ das aus $A, B$ zusammengesetzte System, so ist
\[
	a'b + b'a = S.
\]
\vspace*{-16pt}%
\Beweis
Denn wenn $s$ Element von $a'b$, so ist $s'$ in $b'a$ oder $a'b$
enthalten, je nachdem $s = b$ oder verschieden von $b$ (zufolge (\ref{gthm10})
oder ~(\ref{gthm11}) und (\ref{gdef3}.II)), und ebenso, wenn $s$ Element von
$b'a$, so ist $s'$ in $a'b$ oder $b'a$ enthalten; also ist
$(a'b + b'a)' \TeilVon a'b + b'a$; hieraus folgt der Satz nach \eqref{geq2}.
\end{satz}

\begin{satz}\label{gthm18}
Ist $a$ verschieden von $b$, so ist $ab = [a] + a'b$.
\Beweis
Denn da $a$ ein von $b$ verschiedenes Element von $ab$ ist, so ist $a'$ Element
von $ab$ (\ref{gthm10}, \ref{gthm11}), und folglich (\ref{gthm16}) ist
$a'b \TeilVon ab$; da ferner (\ref{gthm10}) auch $[a] \TeilVon ab$, mithin
\[
	[a] + a'b \TeilVon ab.
\]
Ferner: jedes von $b$ verschiedene Element $s$ von $[a] + a'b$ ist entweder
$= a$ oder ein von $b$ verschiedenes Element von $a'b$, in beiden Fällen ist
$s'$ (nach (\ref{gthm10}), (\ref{gthm11})) Element von $a'b$, also auch von
$[a]+ a'b$, und da (\ref{gthm11}) auch $[b] \TeilVon [a] + a'b$, so ist
$[a] + a'b$ ein System $H_b$; da endlich auch $[a] \TeilVon [a] + a'b$, so ist
(\ref{gthm12}) auch
\[
	ab \TeilVon [a] + a' b.
\]
Aus der Vergleichung beider Resultate folgt der Satz.
\end{satz}

\newpage

\begin{satz}\label{gthm19}
Sind $a, b$ verschiedene Elemente von $S$, so liegt $a$ außerhalb $a'b$,
und $b$ liegt außerhalb $b'a$.
\Beweis
Nimmt man nämlich das Gegenteil an, es gebe ein von $b$ verschiedenes
Element $a$, das in $a'b$ liegt, und bezeichnet mit $A$ das System aller
solcher Elemente $a$, so ergibt sich folgendes.

Setzt man $a' = s$, so liegt $a$ in $sb$, und da $a$ verschieden von $b$ ist, also
(nach (\ref{gthm13})) nicht in $bb$ liegt, so ist $s$ verschieden von $b$, und
hieraus folgt (nach (\ref{gthm18})), dass $sb = [s] + s'b$ ist. Da ferner $a$
(nach (\ref{gthm2})) verschieden von $s$ ist und in $sb$ liegt, so muss $a$ in
$s'b$ liegen, und hieraus folgt wieder (nach (\ref{gthm1})), dass auch $s$
(als Bild $a'$) in $s'b$ liegt.

Mithin ist das Bild $a'$ eines jeden Elementes $a$ von $A$ ebenfalls in $A$
enthalten, also $A' \TeilVon A$. Da aber hieraus $a = S$ folgen würde, während
doch $A$ das Element $b$ nicht enthält, so ist unsere Annahme unzulässig, also
der Satz wahr, w.z.b.w.

Der zweite Teil folgt durch Vertauschung von $a$ mit $b$.%
\end{satz}

\begin{satz}\label{gthm20}
Sind $a, b$ verschieden, so haben die Strecken $a'b, b'a$ kein gemeinsames
Element.
\Beweis
Nimmt man nämlich das Gegenteil an, es gebe ein gemeinsames Element
$m\,$ von $a'b\,$, $b'a\,$, so folgt aus dem vorhergehenden Satz
(\ref{gthm19}~), dass $m$ verschieden von $b$ und von $a$ ist; mithin muss
(\ref{gthm11}) das Bild $m'$ ebenfalls gemeinsames Element von $a'b$ und $b'a$ sein.

Bezeichnet man daher
mit $M$ das System aller solcher Elemente $m$, so ist $M' \TeilVon M$, also
$M = S$. Dies ist aber unmöglich, weil $a, b$ Elemente von $S$, aber nicht
Elemente von $M$ sind. Also ist unser Satz wahr.
\end{satz}

\begin{satz}\label{gthm21}
Sind $a, b$ verschieden, so sind auch die Bilder $a', b'$ verschieden.
\Beweis
Denn sonst hätten die Strecken $a'b, b'a$ ein gemeinsames Element $a' = b'$, weil
$a'$ (nach (\ref{thm10})) Element von $a'b$ und $b'$ Element von $b'a$ ist.
\end{satz}

\begin{satz}\label{gthm22}
Aus $cb = S$ folgt $c = b'$.
\Beweis
Es gibt (nach (\ref{thm1}) und (\ref{thm21})) in $S$ ein und nur ein Element $a$,
welches der Bedingung $a' = c$ genügt, und es ist also $a'b = S$, mithin
$[a] \TeilVon a'b$; es muss daher (\ref{gthm19}) $a = b$, also $c = b'$ sein, w.z.b.w.
\end{satz}

\begin{satz}\label{gthm23}
Sind $a, b$ verschieden, so ist jedes Element von $S$ in einer und nur einer
der Strecken $a'b, b'a$ enthalten.
\Beweis
Dies folgt aus (\ref{thm17}) und (\ref{thm20}).
\end{satz}

\begin{satz}\label{gthm24}
Sind $a,\, b,\, c$ verschieden, so haben die Strecken $b'c,\, c'a$, $a'b$ kein
gemeinsames Element, und dasselbe gilt von den Strecken $a'c,\, b'a,\, c'b$.
\Beweis
Denn die gegenteilige Annahme, es gebe ein den Strecken $b'c,\, c'a,\, a'b$
gemeinsames Element $m$, führt zu einem Widerspruch. Es sei $M$ das System aller
solcher Elemente. Da (nach (\ref{gthm19})) $a$ nicht in $a'b$, $ b$ nicht in
$b'c$, $c$ nicht in $c'a$ liegt, so ist $m$ verschieden von $c, a, b$, und folglich
(\ref{thm11}) ist $m'$ ebenfalls gemeinsames Element von $b'c,\, c'a,\, a'b$,
also Element von $M$.

Mithin ist $M' \TeilVon M$, also $M = S$. Dies ist aber unmöglich, weil $M$ keins
der Elemente $a,\, b,\, c$ enthält. Also ist unser Satz wahr.

Der zweite Teil ergibt sich aus dem ersten, wenn man $a$ mit $b$ vertauscht,
wodurch die Annahme nicht geändert wird.
\end{satz}

\begin{zusatz}\label{zusatz1}
Setzt man (wie auch in dem folgenden (\ref{gthm25})):
\[
    A = c'b,\ B = a'c,\ C = b'a;\  A_1 = b'c,\ B_1 = c'a,\ C_1 = a'b,
\]
so ist $A - B - C = 0$\footnote{[Dabei bedeutet das Zeichen -- den Durchschnitt.]}
(leer) und $A_1 - B_1 - C_1 = 0$ (leer) und (nach (\ref{gthm17}), (\ref{gthm20})) ist
\begin{gather*}
    S = A + A = B + B_1 = C + C_1; \\
    \,0 = A - A_1 = B - B_1 = C - C_1.
\end{gather*}
Dies gilt auch dann (nach (\ref{gthm20})), wenn von den Elementen $a$, $b$, $c$
wenigstens zwei verschieden sind.
\end{zusatz}

\begin{satz}\label{gthm25}
Sind $a$, $b$, $c$ verschieden, so tritt einer und nur einer der beiden folgenden
Fälle ein: Entweder ist
\begin{gather*}
    b'c = b'a + a'c, \quad c'a = c'b + b'a, \quad a'b = a'c + c'b \\
    c'b = c'a - a'b, \quad a'c = a'b - b'c, \quad b'a = b'c - c'a
\end{gather*}
und jedes Element von $S$ liegt in einer, aber nur einer der Strecken
$c'b$, $a'c$, $b'a$; oder es ist
\begin{gather*}
    c'b = c'a + a'b, \quad a'c = a'b + b'c, \quad b'a = b'c + c'a \\
    b'c = b'a - a'c, \quad c'a = c'b - b'a, \quad a'b = a'c - c'b
\end{gather*}
und jedes Element von $S$ liegt in einer, aber nur einer der Strecken
$b'c$, $c'a$, $a'b$.
\Beweis
Zufolge (\ref{gthm23}) liegt $c$ entweder in $a'b$ oder in $b'a$. Wir betrachten
nur den ersten Fall, weil aus ihm der zweite durch Vertauschung von $a$ mit $b$
hervorgeht. Da $c$ in $a'b$ liegt und von $b$ verschieden ist, so liegt
(nach (\ref{gthm11})) auch $c'$ in $a'b$, und folglich (\ref{gthm16}) ist
$c'b \TeilVon a'b$; hieraus folgt (nach \ref{gthm19}), dass $c'b$ mit $b'a$ kein
gemeinsames Element hat; nun ist (mit \ref{gthm17}) $a'b + b'a = b'c + c'b$,
mithin $b'a \TeilVon b'c$ und folglich (\ref{gthm11}) liegt $a$ in $b'c$.

Aus der Annahme, dass $c$ in $a'b$ liegt, hat sich also ergeben: $c'b \TeilVon a'b$,
$b'a \TeilVon b'c$, $a$ liegt in $b'c$. Auf dieselbe Weise ergeben sich aus dieser
letzten Folgerung, wenn man $c$, $a$, $b$ in der Annahme resp. durch $a$, $b$, $c$
ersetzt, wieder die Folgerungen $a'c \TeilVon b'c$, $c'b \TeilVon c'a$, $b$ liegt
in $c'a$; und hieraus folgt abermals $b'a \TeilVon c'a$, $a'c \TeilVon a'b$ (und
die erste Annahme: $c$ liegt in $a'b$).
\end{satz}


%%%%%%%%%%%%%%%%%%%%%%%%%%%%%%%%%%%%%%%%%%%%%%%%%%%%%%%%%%%%%%%%%%%%%%%%%%%%%%%%


\switchcolumn
\selectlanguage{english}

% \marginpar{\textit{What are numbers, and what is their purpose?}}%
\noindent First published in the second edition (1893) of the text
\textit{``Was sind und was sollen die Zahlen?''} page XVII, in the form:%

\begin{quote}
A system $S$ is called finite if it can be mapped into itself in such a way that
no proper part of $S$ is mapped into itself; in the opposite case, $S$ is called
an infinite system.
\end{quote}

Pursuing this definition of a finite system $S$ \emph{without using the natural numbers}.

Let $\varphi$ be a mapping of $S$ into itself, which maps no proper part of $S$
into itself. Small Latin letters $a, b \ldots z$ always mean \emph{elements} of
$S$, capital Latin letters $A, B \ldots Z$ mean \emph{parts} of $S$. The images
of $a, A$ generated by $\varphi$ are respectively denoted by $a', A'$.
\ \\

That $A$ is \emph{part of} $B$ is expressed by $A \partof B$. The system consisting
of the elements $a, b, c, \ldots$ is denoted by $[a, b, c, \ldots]$.
\ \\

This gives
\begin{equation}\label{eq1}
				S' \partof S
\end{equation}
and % \marginpar{\textit{The definition of finiteness.}}
\begin{equation}\label{eq2}
		\text{from } A' \partof A \text{ it follows that } A = S.
\end{equation}


\begin{satz}\label{thm1}$S' = S$.
\Beweis
Every element of $S$ is an image of (at least) one element $r$ of $S$. Because
from \eqref{eq1} it follows $(S')' \partof S'$, hence by \eqref{eq2}, our proposition.
\end{satz}
Every system $[s]$ consisting of a single element $s$ is finite because it has
no proper part and is mapped into itself by the identity function. This case is
\emph{excluded} in the following; $S$ means a finite system that does \emph{not}
consist of a single element.
\smallskip

\begin{satz}\label{thm2}
Every element $s$ is different from its image $s'$, in symbols: $s \neq s'$.
\Beweis
Because if $s = s'$, then $[s]' = [s'] = [s] \partof [s]$, so according to
\eqref{eq2}, also $[s] = S$ in contradiction to our assumption about $S$.
\end{satz}

\newpage

\begin{deff}\label{def3}
If $s$ is a certain element of $S$, then $H_s$ shall denote any part of $S$ that
satisfies the following two conditions:
\vspace{8pt} %\ \\%

\begin{enumerate}[I.]
\item $s$ is element of $H_s$, so $[s] \partof H_s$, also
\[
	[s] + H_s = H_s.
\]
\item If $h$ is an element of $H_s$ different from $s$, then $h'$ is also an
element of $H_s$. So if $H \partof H_s$, but $s$ is \emph{not contained} in H,
then $H' \partof H_s$.
\end{enumerate}
\end{deff}

\begin{satz}\label{thm4}
$S$ and $[s]$ are special systems $H_s$, and $[s]$ is the \emph{intersection}
(the common) of all systems $H_s$ corresponding to the element $s$.
\Beweis Obvious. \end{satz}

\begin{satz}\label{thm5}
$H_s = S$ or $H_s$ is a \textit{proper} part of $S$, depending on whether
$s'$ lies in $H_s$ or not.
\Beweis
For if $s'$ lies in $H_s$, then it follows from (\ref{def3}.II) that
$H'_s \partof H_s$, therefore by \eqref{eq2} that $H_s = S$. Conversely,
if $H_s = S$, then $s'$ also lies in $H_s$.
\end{satz}

\begin{satz}\label{thm6}
If $H_s$ is a \emph{proper} part of $S$, then $s'$ is the only element of
$H'_s$ that lies outside $H_s$.
\Beweis
This is because every element $k$ of $H'_s$ is the image $h'$ of at least one
element $h$ in $H$. If $k=h'$ is different from $s'$, then $h$ is also different
from $s$, and consequently by (\ref{def3}.II) $k = h'$ lies in $H_s$, while the
element $s'$ of $H'_s$ by (\ref{thm5}) lies outside $H_s$.
\end{satz}

\begin{satz}\label{thm7}
Every system $H'_s$ is a system $H_{s'}$, that is (by definition 3.):
\begin{enumerate}[I'.]
	\item $s'$ is element of $H'_s$
	\item If $k$ is an element of $H'_s$ that is different from $s'$, then $k'$
    also lies in $H'_s$.
\end{enumerate}
\beweis
The first follows from the fact that $s$ lies in $H_s$, the second from the fact
that $k$ lies in $H_s$ by (\ref{thm6}).
\end{satz}

\begin{satz}\label{thm8}
If $A$, $B$, $C$ \ldots\ are special systems $H_s$ corresponding to the same $s$,
then their intersection $H$ is also a system $H_s$.
\Beweis
Because according to (\ref{def3}.I) $s$ is a common element of $A, B, C, \ldots$,
thus also an element of $H$. If $h$ is an element of $H$ that is different from
$s$, then, by  (\ref{def3}.II), the image $h'$ is an element of $A$, of $B$, of
$C$, \ldots, and therefore also of $H$. $H$ thus fulfills the two conditions I
and II in definition (\ref{def3}) that are characteristic of every $H_s$.
\end{satz}

\begin{deff}\label{def9}
If $a$, $b$ are certain elements of $S$, then the symbol $ab$ (\emph{section} $ab$)
should mean the intersection of all those systems $H_b$ which (such as $S$)
contain the element $a$.%
\end{deff}

\newpage

\begin{satz}\label{thm10}
$a$ is an element of $ab$, i.e., $[a] \partof ab$.
\Beweis
This is because $ab$ is the intersection of all systems $H_b$ in which $a$ lies.
(So a is the \emph{start} of ab.)
\end{satz}

\begin{satz}\label{thm11}
$ab$ is a system $H_b$, i.e. $[b] \partof ab$, and if $s$ is an element of $ab$
different from $b$, then $[s'] \partof ab$.
\Beweis
This follows from (\ref{thm8}).
\end{satz}

So $b$ is an element (the \emph{end}) of $ab$. If $H \partof ab$ but $b$ is not
contained in $H$, then $H' \partof ab$.

\begin{satz}\label{thm12}
From $[a] \partof H_b$, follows from $ab \partof H_b$.
\Beweis
Immediate consequence of definition (\ref{def9}).
\end{satz}

\begin{satz}\label{thm13}
$aa = [a]$.
\Beweis
This follows from (\ref{thm4}), because $aa$ is the intersection of all $H_a$
that contain the element $a$ according to (\ref{def3}.I).
\end{satz}

\begin{satz}\label{thm14}
If $b'$ is an element of $ab$, then $ab = S$.
\Beweis
This follows from (\ref{thm11}) and (\ref{thm5}).
\end{satz}

\begin{satz}\label{thm15}
$b' b = S$.
\Beweis
This follows from (\ref{thm14}) and (\ref{thm10}).
\end{satz}

\begin{satz}\label{thm16}
If $c$ is an element of $ab$, then $cb \partof ab$.
\Beweis
This follows from (\ref{thm12}), since $ab$ is an $H_b$ by (\ref{thm11}), that
contains the element $c$.
\end{satz}

\begin{satz}\label{thm17}
If $A + B$ means the system composed of $A, B$, then one has
\[
	a'b + b'a = S.
\]
\vspace*{-18pt}%
\Beweis
Because if $s$ is an element of $a'b$, then $s'$ is contained in $b'a$ or $a'b$,
depending on $s = b$ or different from $b$ (according to (\ref{thm10}) or
(\ref{thm11}) and (\ref{def3}.II)), and likewise if $s$ is an element of $b'a$,
then $s'$ is contained in $a'b$ or $b'a$; therefore $(a'b + b'a)' \partof a'b + b'a$.
This leads to the theorem according to \eqref{eq2}.
\end{satz}

\begin{satz}\label{thm18}
If $a$ is different from $b$, then $ab = [a] + a'b$.
\Beweis
For since $a$ is an element of $ab$ different from $b$, then $a'$ is an element
of $ab$ (by \ref{thm10}, \ref{thm11}), and consequently (by \ref{thm16})
$a'b \partof ab$; since furthermore, by (\ref{thm10}), we also have $[a]
\partof ab$, therefore
\[
	[a] + a'b \partof ab.
\]
Also, every element $s$ of $[a] + a'b$ that is different from $b$ is either
$= a$ or an element of $a'b$ that is different from $b$. Thus in both cases $s'$
is (by (\ref{thm10}), (\ref{thm11})) an element of $a'b$, therefore also of
$[a]+ a'b$, and since by (\ref{thm11}) also $[b] \partof [a] + a'b$, it follows
that $[a] + a'b$ is a system $H_b$. Finally, since $[a] \partof [a] + a'b$, by
(\ref{thm12}) also
\[
	ab \partof [a] + a' b.
\]
The theorem follows from the comparison of both results.
\end{satz}

\newpage

\begin{satz}\label{thm19}
If $a, b$ are different elements of $S$, then $a$ lies outside $a'b$, and $b$ lies
outside $b'a$.
\Beweis
If one assumes the opposite, that there is an element $a$ that is different from
$b$ and lies in $a'b$, and that $A$ denotes the system of all such elements $a$,
the following holds.

If one puts $a'=s$, then $a$ lies in $sb$, and since $a$ is different from $b$,
and therefore (according to (\ref{thm13})) is not in $bb$, then $s$ is different
from $b$, and from this it follows (according to \ref{thm18}) that $sb = [s] + s'b$.
Furthermore, since $a$ (according to (\ref{thm2})) is different from $s$ and lies
in $sb$, then $a$ must lie in $s'b$, and from this it follows (again according to
(\ref{thm1})) that $s$ (as the image $a'$) also lies in $s'b$.%

Therefore, the image $a'$ of every element $a$ of $A$ is also contained in $A$,
i.e. $A' \partof A$. But since $A=S$ would follow from this, while $A$ does not
contain the element $b$, our assumption is inadmissible, so the theorem is true,
qed.

The second part follows by exchanging $a$ with $b$.%
\end{satz}

\begin{satz}\label{thm20}
If $a$, $b$ are different, then the segments $a'b$, $b'a$ have no common element.
\Beweis
If one assumes the opposite, that there is a common element $m$ of $a'b$, $b'a$,
then it follows from the preceding Theorem \ref{thm19} that $m$ is different from
$b$ and from $a$; therefore (according to \ref{thm11}) the image $m'$ must also be
a common element of $a'b$ and $b'a$.%

Therefore, if M denotes the system of all such elements $m$, then $M' \partof M$,
thus $M=S$. But this is impossible because $a$, $b$ are elements of $S$ but not
elements of $M$. So our theorem is true.%
\ \\
\end{satz}

\begin{satz}\label{thm21}
If $a$, $b$ are different, then the images $a'$, $b'$ are also different.
\Beweis
Otherwise the segments $a'b$, $b'a$ would have a common element $a'=b'$, because
(according to \ref{thm10}) $a'$ is an element of $a'b$ and $b'$ is an element
of $b'a$.%
\end{satz}

\begin{satz}\label{thm22}
From $cb=S$ follows $c = b$.
\Beweis
There is (according to \ref{thm1} and \ref{thm21}) in $S$ one and only one element
$a$ which satisfies the condition $a'=c$, and therefore $a'b = S$, therefore
$[a] \partof a'b$; therefore (by \ref{thm19}) $a=b$, thus $c=b'$, qed.%
\end{satz}

\begin{satz}\label{thm23}
If $a$, $b$ are different, then every element of $S$ is contained in one and only
one of the segments $a'b$, $b'a$.
\Beweis
This follows from (\ref{thm17}) and (\ref{thm20}).
\end{satz}

\begin{satz}\label{thm24}
If $a$, $b$, $c$ are different, then the segments $b'c$, $c'a$, $a'b$ have no
common element, and the same applies to the segments $a'c$, $b'a$, $c'b$.
\Beweis
Because the opposite assumption, that there is an element $m$ common to the segments
$b'c$, $c'a$, $a'b$, leads to a contradiction. Let $M$ be the system of all such
elements. Since (according to (\ref{thm19})) $a$ is not in $a'b$, $b$ is not in $b'c$,
$c$ is not in $c'a$, then $m$ is different from $c$, $a$, $b$, and consequently
(by \ref{thm11}) $m'$ is a common element of $b'c$, $c'a$, $a'b$, i.e. an element
of $M$; therefore $M' \partof M$, hence  $M=S$.

But this is impossible because $M$ does not contain any of the elements $a$, $b$,
$c$. So our theorem is true.

The second part results from the first if one swaps $a$ with $b$, which does not
change the assumption.
\end{satz}

\begin{zusatz}\label{corollary1}
If you put (as in the following \ref{thm25}):
\[
    A = c'b,\ B = a'c,\ C = b'a;\  A_1 = b'c,\ B_1 = c'a,\ C_1 = a'b,
\]
then $A - B - C = 0$\footnote{[The symbol -- means the intersection.]} (empty) and
$A_1 - B_1 - C_1 = 0$ (empty) and (according to \ref{thm17}, \ref{thm20}) hence
\begin{gather*}
    S = A + A = B + B_1 = C + C_1; \\
    \,0 = A - A_1 = B - B_1 = C - C_1.
\end{gather*}
This also applies (according to \ref{thm20}) if at least two of the elements
$a$, $b$, $c$ are different.
\end{zusatz}

\begin{satz}\label{thm25}
If $a$, $b$, $c$ are different, then one and only one of the following two
cases occurs: Either
\begin{gather*}
    b'c = b'a + a'c, \quad c'a = c'b + b'a, \quad a'b = a'c + c'b \\
    c'b = c'a - a'b, \quad a'c = a'b - b'c, \quad b'a = b'c - c'a
\end{gather*}
and each element of S lies in one, but only one, of the segments
$c'b$, $a'c$, $b'a$; or
\begin{gather*}
    c'b = c'a + a'b, \quad a'c = a'b + b'c, \quad b'a = b'c + c'a \\
    b'c = b'a - a'c, \quad c'a = c'b - b'a, \quad a'b = a'c - c'b
\end{gather*}
and each element of $S$ lies in one, but only one, of the segments
$b'c$, $c'a$, $a'b$.
\Beweis
According to \ref{thm23}, $c$ lies either in $a'b$ or in $b'a$. We
only consider the first case because the second arises from it by
exchanging $a$ for $b$.
Since $c$ is in $a'b$ and is distinct from $b$, then
(according to \ref{thm11}) $c'$ also lies in $a'b$, and consequently
(by \ref{thm16}) $c'b \partof a'b$; from this it follows (by \ref{thm19})
that $c'b$ has no element in common with $b'a$; now (by \ref{thm17}) is
$a'b+b'a=b'c+c'b$, therefore $b'a \partof b'c$, and consequently
(by \ref{thm11}) $a$ is in $b'c$.

From the assumption that $c$ lies in $a'b$, it follows: $c'b \partof a'b$,
$b'a \partof b'c$, $a$ lies in $b'c$. In the same way, this last
conclusion follows if one assumes $c$, $a$, $b$ replaced by $a$, $b$, $c$,
respectively, again we have the consequences $a'c \partof bc$, $cb \partof c'a$,
and that $b$ lies in $c'a$; and from this it follows again $b'a \partof c'a$,
$a'c \partof a'b$ (and the first assumption: $c$ lies in $a'b$).
\end{satz}

\end{paracol}
\end{document}
