\documentclass[12pt]{article}
\usepackage[ngerman]{babel}
\usepackage[utf8]{inputenc}
\usepackage[T1]{fontenc}
\usepackage{amsmath}
\usepackage{amsfonts}
\usepackage{amssymb}
\usepackage[version=4]{mhchem}
\usepackage{stmaryrd}

% 464 Session of the physical-mathematical class of May 29, 1913.

\title{On Markoff's Numbers}

\author{by G. Frobenius.}
\date{}

\begin{document}
\maketitle
Andrej Markoff, in \textit{Math. Annalen, vols. 15 and 17}, published two works, \textit{Sur les formes quadratiques binaires indéfinies}. In the indefinite form
$$
\psi=(a, b, c)=a x^{2}+b x y+c y^{2}
$$
the coefficients $a, b, c$ are arbitrary real numbers, and the variables $x, y$ are integers. The discriminant of $\psi$ is $D=b^{2}-4 a c$, and the lower bound of all values of the absolute value $|\psi|$ of $\psi$ is $M$. For the form $k \psi$, these quantities have the values $k^{2} D$ and $k M$.  Markoff proves the following.

For all indefinite forms $\psi$,
$$
\text{ lim. inf. } \frac{V \sqrt{D}}{M}=3 .
$$
If $\sqrt{D}<3 M$, then $\psi$, multiplied by a suitable factor $k$, is equivalent (properly or improperly) to a form
$$
\varphi=p x^{2}+(3 p-2 q) x y+(r-3 q) y^{2}
$$
Here $p, q, r$ are positive integers, and $p$, together with two other integers $p_{1}$ and $p_{2}$, satisfies the indeterminate equation
$$
p^{2}+p_{1}^{2}+p_{2}^{2} = 3 p\, p_{1}\, p_{2},
$$
$\pm q$ is the absolute smallest remainder of $\frac{p_{1}}{p_{2}}(\bmod p)$, and $r$ is determined by the equation $p r-q^{2}=1$. For this form $\varphi$, we have
$$
D=9 p^{2}-4, \quad M=p, \quad \frac{\sqrt{D}}{M}=3 \sqrt{1-4: 9 p^{2}}<3 .
$$
The form $\varphi$ is properly equivalent to the form $-\varphi$, and even to each of the four forms
$$
\pm\left(p x^{2}-2 q x y+r y^{2}\right) \pm 3 y(p x-q y) .
$$
If the ratios of the coefficients of $\psi$ are not rational, then $M \leqq \frac{1}{3} \sqrt{D}$ always holds.

Despite these extraordinarily remarkable and important results, these difficult investigations seem to be little known. Even Minkowski does not mention them in his treatment of a related question (\textit{Math. Annalen, vol. 54, p. 92}). To my knowledge, Hurwitz (\textit{Über eine Aufgabe der unbestimmten Analysis, Archiv der Math. u. Phys., series 3, vol. 11, p. 185}) is the only one who has written about the Markoff equation. The large but so far little-used theory of the reduction of indefinite binary quadratic forms, which Lagrange created and Gauss perfected, finds extensive application in the following developments.

Markoff proves the results using continued fractions. I have succeeded (§ 4) in deriving the properties of the form $\varphi$ without this tool, but not in proving that the forms equivalent to the forms $\varphi$ are the only ones for which $\sqrt{D}<3M$. In the second part of my work, I develop the explicit representation of the Markoff numbers $p$ and the corresponding numbers $q$ and $r$ through the partial denominators of a continued fraction. This leads to (§ 11) remarkable relationships with the concept of the characteristic of a rational number created by Christoffel (\textit{Lehrsätze über arithmetische Eigenschaften der Irrationalzahlen, Annali di Mat., ser. II, vol. 15, p. 270}).

I will refer to this work as C., Markoff's second work (\textit{Math. Annalen, vol. 17, p. 379}) as M., Hurwitz's work as H., and my work (\textit{Über die Reduktion der indefiniten binären Formen, in these proceedings, p. 202}) as F.

\bigskip

\centerline{§ 1.}

\bigskip

I call the indeterminate equation

\begin{equation*}
a^{2}+b^{2}+c^{2}=3 a b c \tag{1.}
\end{equation*}
the \textit{Markoff equation}, and any positive integer $p$ that appears in one of its solutions a \textit{Markoff number}.

\bigskip

\textit{I. A positive integer $p$ is called a Markoff number if $-p^{2}$ can be represented by the principal form of the discriminant $9 p^{2}-4$.}

\bigskip

Following Mr. Hurwitz, I first consider the more general equation
\begin{equation*}
a^{2}+b^{2}+c^{2}=k \,a\, b\, c \tag{2.}
\end{equation*}
where all signs represent positive integers.
Depending on whether $a$ is even or odd, $a^{2} \equiv 0$ or $1(\bmod 4)$. So, if $k$ is even, $a, b, c$ must all be even. Any other assumption about their remainders $(\bmod 2)$ leads to a contradiction. However, if $a=2^{n} x, b=2^{n} y, c=2^{n} z$, where $x, y, z$ are not all even, then $x^{2}+y^{2}+z^{2}=2^{n} k x y z$. Since this equation requires that $x, y, z$ be all even, it follows that $k$ cannot be even.

Depending on whether $a$ is divisible by 3 or not, $a^{2}=0$ or 1 $(\bmod 3)$. So if $k$ is not divisible by 3, $a=3 x, b=3 y, c=3 z$ must all be divisible by 3. Any other assumption proves to be inadmissible. Then $x^{2}+y^{2}+z^{2}=3 k x y z$.

After dealing thus with the case $k=2$, and reducing $k=1$ to $k=3$, let $k \geq 3$. If $b=c$, then $a=b d$ is divisible by $b$ and $d^{2}+2=k b d$. Therefore, $d=1$ or 2, a divisor of 2. In both cases, $3=k b$, so $k=3, b=c=1, a=d$. So, if $k>3$, no two of the numbers $a, b, c$ can be equal. For $k=3$, I will call the two solutions $1,1,1$ and $2,1,1$ \textit{singular}. In every other solution, $a, b, c$ are different. Many of the results to be developed do not apply to these two solutions, which I will not always mention specifically.

If the equation $f(x)=x^{2}+b^{2}+c^{2}-k b c x=0$ has the two roots $a$ and $a^{\prime}$, then $a+a^{\prime}=k b c$, $a a^{\prime}=b^{2}+c^{2}$. So $a^{\prime}$ is also a positive integer and $a^{\prime}, b, c$ is a new solution of equation (2.). Mr. Hurwitz calls it a neighboring solution to the solution $a, b, c$. If $a>b>c$, then $f(b)<3 b^{2}-k c b^{2} \leq 0$. Therefore $b$ lies between the two roots $a$ and $a^{\prime}$, and we have $a>b>a^{\prime}$. If the product $a b c$ is called the weight of the solution $a, b, c$, then the new solution therefore has a smaller weight than the original. This procedure of forming new solutions of smaller weight can always be continued if the three numbers of the solution are different. Consequently, $k$ cannot be greater than 3. However, if $k=3$, it must eventually lead to a singular solution.

Therefore, equation (2.) is only solvable for $k=3$ and $k=1$, and the second case can be reduced to the first by substituting $a=3 x, b=3 y, c=3 z$.

Three numbers that satisfy equation (I.) have no common divisor (H. p. 194). For if $a=d x, b=d y, c=d z$, then $x^{2}+y^{2}+z^{2}=3 d x y z$, and therefore $d=1$. Consequently, any two of the numbers $a, b, c$ are also coprime, and the representation mentioned in I. is always a proper one. In particular, at most one of them can be even. However, $a$ cannot be divisible by 4. Otherwise, $0 \equiv 3 a b c=a^{2}+b^{2}+c^{2}=0+1+1(\bmod 4)$. Since
\begin{equation*}
a+a^{\prime}=3 b c, \quad a a^{\prime}=b^{2}+c^{2} \tag{3.}
\end{equation*}
every odd prime that divides $a$ must be of the form $4n+1$. This implies:

\bigskip

II. \textit{If $p$ is a Markoff number, then either $p \equiv 1(\bmod 4)$ or $p \equiv 2(\bmod 8)$.}

\bigskip

From the equations
$$
a^{2}+(b-c)^{2}=b c(3 a-2), \quad a^{2}+(b+c)^{2}=b c(3 a+2)
$$
it follows:

\bigskip

III. \textit{If $p$ is a Markoff number, then every odd prime factor of $p, 3p-2$ and $3p+2$ is of the form $4n+1$.}

\bigskip

However, numbers like 37 or 61, which are not Markoff numbers, also possess this property.

The singular solution 1,1,1 has only one neighboring solution 2, 1, 1. This one has, in addition to that, a second 5,2,1. Any other solution $a, b, c$ has three different neighboring solutions (M. p. 397)
$$a^{\prime}, b, c \quad a, b^{\prime}, c \quad a, b, c^{\prime},$$
where
$$
(4.) \quad a^{\prime}=3 b c-a, \quad b^{\prime}=3 a c-b, \quad c^{\prime}=3 a b-c.
$$
If $a$ is the largest of the three numbers $a, b, c$, then
$$
a^{\prime}<a, \quad b^{\prime}>a, \quad c^{\prime}>a
$$
and even $a^{\prime}$ is smaller than the larger of the two numbers $b$ and $c$. Of the three neighboring solutions $a, b, c$, one, $a^{\prime}, b, c$, has a smaller weight, and each of the other two has a larger weight. For a solution $L$, there is one and only one neighboring solution $L_{1}$ of smaller weight, and for this one, there is such a solution $L_{2}$, and so on. The series of these solutions $L, L_{1}, L_{2}, \cdots$ must, according to the above statements, end with $5,2,1 \quad 2,1,1 \quad 1,1,1$. In the reverse way, one arrives from 1,1,1 at every solution. But if one wants to ascend from a solution L to solutions of a higher weight, this can be done in two different ways at each point.

When applying this procedure to a given solution $p, a, b$, the restriction of keeping the number $p$ fixed can be imposed. I will call all the solutions obtained this way a \textit{chain of solutions} (cf.  9). It is completely determined by each of its members. It contains one and only one solution in which $p$ is the largest of the three numbers. The question of whether a Markoff number $p$ can correspond to two different chains, i.e., whether $p$ can be the largest number in two solutions $p, a, b$ and $p, c, d$, is to date undecided.

From (3.) it follows
$$
\left(a-a^{\prime}\right)^{2}=9 b^{2} c^{2}-4\left(b^{2}+c^{2}\right)=4 b^{2} c^{2}+\left(b^{2} c^{2}-4\right)+4\left(b^{2}-1\right)\left(c^{2}-1\right) .
$$
If $a$ is the largest of the numbers $a, b, c$, and $b$ and $c$ are not both 1, then (cf. (3.))
$$
a-a^{\prime}>2 b c, \quad 3 b c>a>2 b c .
$$
If $b>c$, then in the solution $a^{\prime}, b, c$, the largest number is $b$. Therefore $b>2a^{\prime}c$ and

\begin{equation*}
a^{\prime}<\frac{b}{2 c}, \quad a>3 b c-\frac{b}{2 c}>\frac{5}{2} b c \quad(b>c), \tag{6.}
\end{equation*}
and also
\begin{equation*}
b>\frac{5}{2} a^{\prime} c>\frac{5}{2} c . \tag{7.}
\end{equation*}

Apart from the singular solutions, the neighboring solution 5,2,1 also forms an exception.

\bigskip

\centerline{§ 2.}

\bigskip


From the relation
$$
p^{2}+p_{1}^{2}+p_{2}^{2}=3 p p_{1} p_{2}
$$
we have concluded that any two of the three positive integers $p_{\kappa}$ are coprime. Therefore,

\begin{equation*}
\frac{p_{1}}{p_{2}} \equiv-\frac{p_{2}}{p_{1}} \quad(\bmod p) \tag{1.}
\end{equation*}
Let $\varepsilon= \pm 1$, and let
\begin{align*}
\varepsilon q \equiv \frac{p_{1}}{p_{2}} \equiv-\frac{p_{2}}{p_{1}} & (\bmod p), \\
\varepsilon q_{1} \equiv \frac{p_{2}}{p} \equiv-\frac{p}{p_{2}} & \left(\bmod p_{1}\right),  \tag{2.}\\
\varepsilon q_{2} \equiv \frac{p}{p_{1}} \equiv-\frac{p_{1}}{p} & \left(\bmod p_{2}\right),
\end{align*}

where $q_{\kappa}$ is between 1 and $p_{\kappa}-1$. Then
$$
p_{1}^{2}\left(1+q^{2}\right) \equiv p_{1}^{2}+p_{2}^{2} \equiv 0 \quad(\bmod p)
$$
Therefore, the equations
\begin{equation*}
\quad p r-q^{2}=1, \quad p_{1} r_{1}-q_{1}^{2}=1, \quad p_{2} r_{2}-q_{2}^{2}=1
\tag{3}
\end{equation*}
determine three integers $r_{\kappa}$. For example (cf. § 5)

\bigskip

\begin{tabular}{rrrrrrrrrrrrr}
$r=1$ & 2 & 5 & 13 & 29 & 34 & 89 & 169 & 194 & 233 & 433 & 610 & 985 \\
$q=0$ & 1 & 2 & 5 & 12 & 13 & 34 & 70 & 75 & 89 & 179 & 233 & 408 \\
$r=1$ & 1 & 1 & 2 & 5 & 5 & 13 & 29 & 29 & 34 & 74 & 89 & 169.
\end{tabular}

\bigskip

Now, if $p$ is the largest of the three numbers $p_{\kappa}$, then

\begin{align*}
& p_{1} q_{2}-p_{2} \dot{q}_{1}=\varepsilon\left(p-3 p_{1} p_{2}\right)=-\varepsilon p^{\prime}, \\
& p_{2} q-p q_{2}=\varepsilon p_{1},  \tag{4.}\\
& p q_{1}-p_{1} q=\varepsilon p_{2} .
\end{align*}

For $p_{1} q_{2}-p_{2} q_{1}+\varepsilon p^{\prime}$ is divisible by $p_{1}$ and $p_{2}$, thus by $p_{1} p_{2}$, and is $\leqq p_{1}\left(p_{2}-1\right)-p_{2}+p^{\prime}<p_{1} p_{2}$, because $p^{\prime}$ is smaller than the larger of the two numbers $p_{1}$ and $p_{2}$. On the other hand, that number is $\geqq p_{1}-p_{2}\left(p_{1}-1\right)-p^{\prime} >-p_{1} p_{2}$.

By adding equations (4.) multiplied by $p, p_{1}, p_{2}$ or by $q, q_{1}, q_{2}$ or by adding their squares, one arrives at the following relations:

\begin{align*}
p^{2}+p_{1}^{2}+p_{2}^{2} & =3 p p_{1} p_{2} \\
p q+p_{1} q_{1}+p_{2} q_{2} & =3 q p_{1} p_{2} \\
q^{2}+q_{1}^{2}+q_{2}^{2} & =3 r p_{1} p_{2}-1 \\
p r+p_{1} r_{1}+p_{2} r_{2} & =3 r p_{1} p_{2}+2  \tag{5.}\\
\frac{1}{3}\left(q r+q_{1} r_{1}+q_{2} r_{2}\right) & =p_{1} q_{2} r+\varepsilon q_{1}^{2}=p_{2} q_{1} r-\varepsilon q_{2}^{2} \\
\frac{1}{3}\left(r^{2}+r_{1}^{2}+r_{2}^{2}\right) & =-p r_{1} r_{2}+p_{1} r r_{2}+p_{2} r r_{1}+3
\end{align*}

\medskip

The last two, which are not used here, have only been added for the sake of completeness.

If we set
$$U_{x}=p_{x} u+q_{x} v+r_{x} w,$$
we can summarize the 6 relations in the identical equation
\begin{equation*}
U^{2}+U_{1}^{2}+U_{2}^{2}+3 p U_{1} U_{2}-3 p_{1} U I_{2}-3 p_{2} U U_{1}=4 u w-v^{2}+9 w^{2}  \tag{6}
\end{equation*}
If we set

\begin{equation*}
P_{x}=\mu_{x} x^{2}-2 \eta_{x} x y+r_{x} y^{2} , \tag{7.}
\end{equation*}
it is therefore
\begin{equation*}
P^{2}+P_{1}^{2}+P_{2}^{2}+3 p P_{1} P_{2}-3 p_{1} P P_{2}-3 p_{2} P P_{1}=9 y^{4} . \tag{8.}
\end{equation*}

The first of relations (5.) can also be written in the form
\begin{equation*}    
p_{1}^{2}+p_{2}^{2}=p p^{\prime}, \quad p_{1} q_{1}+p_{2} q_{2}=q p^{\prime} , \quad p_{1} r_{1}+p_{2} r_{2}=\cdots r p^{\prime}+2,  \tag{9}
\end{equation*}
or summarized
\begin{equation*}
p_{1} P_{1}+p_{2} P_{3}=p^{\prime} P+2 y^{2}, \tag{10}
\end{equation*}
where $p^{\prime}==3 p_{1} p_{2}-p$. If $p>p_{1}>p_{2}$, the analogous formula for the neighboring solution $p_{1}, p_{2}, p^{\prime}$ of the Markoff equation is
\begin{equation*}
p_{\mathrm{a}} P_{\mathrm{a}}+p^{\prime} P^{\prime}=\left(3 p_{2} p^{\prime}-p_{1}\right) P_{1}+2 y^{2} . 
\end{equation*}
From these two equations, it follows that
\begin{equation*}
P^{\prime}=3 p_{1} P_{1}-P,  \quad \left(p>p_{1}>p_{2}\right) \tag{11}
\end{equation*}
or
\begin{equation*}
p^{\prime}=3 p_{1} p_{2}-p, \quad q^{\prime}=3 p_{3} q_{1}-q, \quad r^{\prime}=3 p_{2} r_{1}-r. \tag{12.}
\end{equation*}
The third equation (9.) in connection with the relations (4.) leads to the formula
\begin{equation*}
\left|\begin{array}{lll}
p & p_{1} & p_{2}  \\
q & q_{1} & q_{2} \\
r & r_{1} & r_{2}
\end{array}\right|=2 \varepsilon.  \quad \tag{13.}
\end{equation*}
If we square the equations (4.), we get, using the formulas (3.),
\begin{align*}
& p_{1} r_{2}+p_{2} r_{1}-2 q_{1} q_{2}=3 p^{\prime}, \\
& p_{2} r+p r_{2}-2 q_{2} q=3 p_{1},  \tag{14.}\\
& p r_{1}+p_{1} r-2 q q_{1}=3 p_{2}.
\end{align*}
Thus, the discriminant of the quadratic form 
$u P+v P_{1}+w P_{2}$ of the variables $x$ and $y$ is equal to -4 times
\begin{gather*}
\left(p u+p_{1} v+p_{2} w\right)\left(r u+r_{1} v+r_{2} w\right)-\left(q u+q_{1} v+g_{2} w\right)^{2}  \tag{15.}\\
=u^{2}+v^{2}+w^{2}+3\left(p^{\prime} v w+p_{1} w u+p_{2} u v\right)
\end{gather*}
This ternary form of $u, v, w$ (cf. (6.)) therefore has a determinant of 1, according to (13.). The Markoff equation
$$
a^{2}+b^{2}+c^{2}=3 a b c
$$
thus states that the ternary quadratic form
\begin{equation*}
u^{2}+v^{2}+w^{2}+3(a v w+b w u+c u v) \tag{16.}
\end{equation*}
has a determinant of 1.
If we write relations (4.) as homogeneous linear equations between $p, p_{1}, p_{2}$, we can determine the ratios of these unknowns from any two of them. For example,
\begin{align*}
\varepsilon p-q_{2} p_{1}+\left(q_{1}-3 \varepsilon p_{1}\right) p_{2} &= 0, \\
q_{2}p + \varepsilon p_{1} - q p_{2} &= 0,
\end{align*}
and therefore
\begin{equation*}
p: p_{1}: p_{2}=q q_{2}-\varepsilon\left(q_{1}-3 \varepsilon p_{1}\right): q_{2}\left(q_{1}-3 \varepsilon p_{1}\right)+\varepsilon q: p_{2} r_{2} .
\end{equation*}

In this way, we arrive at the formulas
\begin{align*}
p_{1} r-q q_{1}=-\varepsilon q_{2},\ & p r_{1}-q q_{1}=\varepsilon q_{2}+3 p_{2}, \\
p_{2} r-q q_{2}=+\varepsilon q_{1},\ & p r_{2}-q q_{2}=-\varepsilon q_{1}+3 p_{1},  \tag{17.}\\
p_{1} r_{2}-q_{1} q_{2}=+\varepsilon\left(q-3 p_{1} q_{2}\right),\ & p_{2} r_{1}-q_{1} q_{2}=-\varepsilon\left(q-3 p_{2} q_{1}\right) .
\end{align*}

Now we can also calculate the subdeterminants of (13.). For example,
$$ (18.) q r_{1}-q_{1} r=\varepsilon r_{2}+3 q_{2}, \quad y_{2} r-q r_{2}=\varepsilon r_{1}-3 q_{1}, \quad q_{1} r_{2}-q_{2} r_{1}=\varepsilon\left(r-3 q_{1} q_{3}\right).
$$

\bigskip

\centerline{§ 3.}

\bigskip

The forms $P_{1}$ and $P_{2}$ have the discriminant -4. Their simultaneous invariant is positive:
\begin{equation*}
\quad p_{1} r_{2}-2 q_{1} q_{2}+r_{1} p_{2}=-3 \varepsilon\left(p_{1} q_{2}-p_{2} q_{2}\right)  \tag{1.}
\end{equation*}
or
\begin{equation*}
p_{1} r_{2}-2 q_{1} q_{2}+r_{1} p_{2}=3\left|p_{1} q_{2}-p_{2} q_{1}\right| . \tag{2.}
\end{equation*}

\end{document}